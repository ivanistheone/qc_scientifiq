






The momentum is equal to the velocity of the moving object multiplied by the object's mass
 ($\vec{p} = m\vec{v}$).
Therefore, since the car weights $1000\times1000=10^{6}$ times more than the piece of paper, it has $10^6$ times more momentum when moving at the same speed. 
A collision with it will ``hurt'' that much more.

%Note that momentum is a vector, so we will have to do a lot of that length-and-direction-to-components transformation stuff:
%\[
% (p_x, p_y) = \vec{p} = (|\vec{p}|\cos\theta, |\vec{p}|\sin\theta) = |\vec{p}| \angle \theta,
%\]
%and also converting backwards from component notation to magnitude-direction:
%\[
% |\vec{p}| = \sqrt{ p_x^2 + p_y^2 }, \qquad \theta = \tan^{-1}\!\left( \frac{ p_{y} }{ p_{x} } \right).
%\]


\subsection{Concepts}
\label{ff4e01de0bd379280a0157bd102cc5f0}%% concepts

\begin{itemize}
\dokuitem  $m$: the \dokuitalic{mass} of the moving object.
\dokuitem  $\vec{v}$: the \dokuitalic{velocity} of the moving object.
\dokuitem  $\vec{p}=m\vec{v}$: the \dokuitalic{momentum} of the moving object.
\end{itemize}

\subsection{Formulas}
\label{51d24e1edefe34e683025dbba5c6eed6}%% formulas

\subsubsection{Definition}

The momentum of an object is the mass of the object times its velocity:
\[
 \vec{p} = m\vec{v}.
\]

If you speed is $\vec{v}=(20,0,0)$[m/s], which is 
equivalent to saying ``20[m/s] in the $x$-axis direction'', and your mass is 100kg then your momentum is $\vec{p}=(2000,0,0)$[kg*m/s].


%\subsubsection{Newton's first law}

%Newton's first law states that whenever there is no acceleration ($\vec{a}=0$), 
%an object will maintain a constant velocity. 
%This is kind of obvious if you know Calculus, since $\vec{a}$ is the derivative of $\vec{v}$.
%For example, if an object is stationary and there are no forces on it to cause it to accelerate, then it will stay stationary.
%If an object was moving with velocity $\vec{v}$ and there is no acceleration, then it will keep moving with velocity $\vec{v}$ forever.
In the absence of acceleration, objects will conserve their velocity:
\[
  \vec{v}_{in}= \vec{v}_{out}.
\]
This is equivalent to saying that objects conserve their momentum (just multiply the velocity by the mass  if the mass stays constant and the velocity stays constant, then the momentum must stay constant).


\subsubsection{Conservation of momentum}

More generally, if you have a situation with multiple moving objects,
you can say that the ``overall momentum'', i.e., the sum of the momenta
of all the particles stays constant:
\[
 \sum \vec{p}_{in} =  \sum \vec{p}_{out}.
\]

This is amazingly powerful stuff, and one of the furthest reaching 
laws of physics. Whatever momentum comes into a collision must come out.
%This is a kind of Le Chatelier's principle 
%``rien ne se cr�e, et rien ne disparait tout se transforme''.
%Motion cannot simply appear or disappear, it can only be exchanged between systems.





\section{Energy}
\label{05e7d19a6d002118deef70d21ff4226e}%% energy

Instead of thinking about velocities $v(t)$ and motion trajectories $x(t)$, we can solve physics problems using \dokuitalic{energy} calculations.
%The key idea in this section is the principle \dokuitalic{total energy conservation}, which tells us
%that the sum of the initial energies is equal to the sum of the final energies.
%We will define precisely the different kinds of energy, and then learn the rules of converting one energy into another.

\vspace{-3mm}
\subsection{Concepts}
\label{ff4e01de0bd379280a0157bd102cc5f0}%% concepts




%and it hits the ground at exactly $t=3$[s].
%What was the acceleration of the ball during its flight?

%
%We know that this was a motion \dokuitalic{with} acceleration, so we start
%by writing out the general UAM equations:
%\begin{align*}
% y(t) &= \frac{1}{2}at^2 + v_i t + y_i, \\
% v(t) &= at + v_i. 
%\end{align*}

%To find the answer, we substitute the known values
%$y(0)=y_i=44.145$[m], $y(3)=0$[m] and $v_i=0$[m/s] (since the ball
%was released from rest) and solve for $a$ in the resulting equation:
%\[
% y(3) = 0 = \frac{1}{2}(a)3^2+0(3) + 44.145.
%\]
%The answer is $a=\frac{-2 \times 44.145}{9}=-9.81$[m/s\^{ }2].

%
%\dokutitlelevelfour{0 to 100 in 5 seconds}

%Suppose you want to be able to go from $0$ to $100$[km/h] in $5$ seconds with your car.
%How much acceleration does your engine need to produce,
%assuming it produces a constant amount of acceleration.

%We can calculate the necessary $a$ by plugging the required values into the velocity equation for UAM:
%\[
% v(t) = at + v_i.
%\]
%Before we get to that, we need to convert 
%the velocity in [km/h] to velocity in [m/s]: 
%$100$[km/h] = /
%$\frac{100 [km]}{1 [h]}\cdot\frac{1000[m]}{1[km]}\cdot\frac{1[h]}{3600[s]}$= 27.8 [m/s]. 
%We fill in the equation with all the desired values $v(5)=27.8$, $v_i=0$, and $t=5$
%and solve for $a$:
%\[
% v(5) = 27.8 = a(5) + 0.
%\]
%We conclude that your engine will have to produce
%a constant acceleration of $a=5.56$[m/s\^{ }2] or more.

%
%\dokutitlelevelfour{Moroccan example II}

%Some time later, your friend wants to send you another aluminum 
%ball from 