%and it hits the ground at exactly $t=3$[s].
%What was the acceleration of the ball during its flight?

%
%We know that this was a motion \dokuitalic{with} acceleration, so we start
%by writing out the general UAM equations:
%\begin{align*}
% y(t) &= \frac{1}{2}at^2 + v_i t + y_i, \\
% v(t) &= at + v_i. 
%\end{align*}

%To find the answer, we substitute the known values
%$y(0)=y_i=44.145$[m], $y(3)=0$[m] and $v_i=0$[m/s] (since the ball
%was released from rest) and solve for $a$ in the resulting equation:
%\[
% y(3) = 0 = \frac{1}{2}(a)3^2+0(3) + 44.145.
%\]
%The answer is $a=\frac{-2 \times 44.145}{9}=-9.81$[m/s\^{ }2].

%
%\dokutitlelevelfour{0 to 100 in 5 seconds}

%Suppose you want to be able to go from $0$ to $100$[km/h] in $5$ seconds with your car.
%How much acceleration does your engine need to produce,
%assuming it produces a constant amount of acceleration.

%We can calculate the necessary $a$ by plugging the required values into the velocity equation for UAM:
%\[
% v(t) = at + v_i.
%\]
%Before we get to that, we need to convert 
%the velocity in [km/h] to velocity in [m/s]: 
%$100$[km/h] = /
%$\frac{100 [km]}{1 [h]}\cdot\frac{1000[m]}{1[km]}\cdot\frac{1[h]}{3600[s]}$= 27.8 [m/s]. 
%We fill in the equation with all the desired values $v(5)=27.8$, $v_i=0$, and $t=5$
%and solve for $a$:
%\[
% v(5) = 27.8 = a(5) + 0.
%\]
%We conclude that your engine will have to produce
%a constant acceleration of $a=5.56$[m/s\^{ }2] or more.

%
%\dokutitlelevelfour{Moroccan example II}

%Some time later, your friend wants to send you another aluminum 
%ball from 