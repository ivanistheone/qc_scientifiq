
To solve a physics problem is to obtain the \emph{equation of motion} $x(t)$, 
which describes position of the object as a function of time.
%
Once you know $x(t)$, you can answer many of the question pertaining to the motion of the object.
To find the initial position $x_i$ of the object, you simply plug $t=0$ into the equation of motion $x_i = x(0)$.
To find the time(s) when the object reaches a distance of 20[m] from the origin, we simply solve for $t$ in $x(t)=20$[m].
Many of the problems on you final exam in physics will be of this form, so it is really important that you know
how to find the equation of motion for any object.

%%\paragraph{{\bf Forces}}
The first step in finding $x(t)$ is to calculate all the \emph{forces} that act on the object.
Forces are the \emph{cause} of motion, so if you want to understand motion you need to understand forces. 
Newton's second law $\sum \vec{F} = \vec{F}_{net}=m\vec{a}$ states that forces acting on an object 
produce an \emph{acceleration} inversely proportional to the mass of the object. 
%Once you have the acceleration, you can compute $x(t)$ using two simple calculus steps.
%For now, though, we want to focus on the causes of motion: the forces acting on the object.
%There are many kinds of forces: the weight of an object $\vec{W}$ is a type force, 
%the force of friction $\vec{F}_f$ is another type another, the tension in a rope $\vec{T}$ yet another type of force
%and there are many others.
%Note the little arrow on top of each force, which is there to remind you that forces are \emph{vector} quantities.
%Unlike regular numbers, forces act in a particular direction, so it is possible that the effects of 
%one force are counteracting another force. For example the force of the weight of a flower pot 
%is exactly counter-acted by the tension in the rope on which it is suspended, thus,
%while there are two forces that may be acting on the pot, there is no \emph{net force} acting on it.
%Since there is no net force to cause motion and since the pot wasn't moving to begin with, 
%it will just sit there motionless despite the fact that there are forces acting on it!
%The first step when analyzing a physics problem will be calculation of 
%acting on the object, which 
The \emph{net force} on an object is the sum of all the forces acting on the object: 
$\vec{F}_{net} \equiv \sum \vec{F}$. Once you have $\vec{F}_{net}$ you know its 
acceleration using $\vec{a}=\frac{\vec{F}_{net}}{m}$.

It turns out that once you know the acceleration of an object $a(t)$, 
you can easily find its velocity $v(t)$ function and once you know the velocity 
function you can find the position function $x(t)$.
%The acceleration is the change in the velocity of the object, thus if you know
%that the object stated with an initial velocity of $v_i \equiv v(0)$,
%and you want to find the velocity at later time $t=\tau$, 
%you have to add up all the acceleration that the object felt during this time $v(\tau)=v_i+\int_0^\tau a(t)\;dt$.
\begin{equation}
 \frac{1}{m}\sum \vec{F} = a(t) \ \overset{v_i+ \int\!dt }{\longrightarrow} \ v(t) \ \overset{x_i+ \int\!dt }{\longrightarrow} \ x(t).
 \label{fma-eqn}
\end{equation}
The symbol $\int \cdot \;dt$ is called an \emph{integral} and is fancy way of finding the total
of some quantity over time. 

initial velocity of $v_i \equiv v(0)$,

initial position $x_i$

%The right hand side of the equation is called a \emph{dynamics} problem and involves 
%the calculation of the \emph{net force} $\vec{F}_{net} \equiv \sum \vec{F}$.
%The right hand side in the above equation is called \emph{kinematics} and focusses on
%the use of integration in order to find $v(t)$ from $a(t)$ and $x(t)$ from $v(t)$.
%Don't worry about this integration business; 
%it is quite simple and we will cover everything you need to know about it in the next section.  


%Newton's second law can also be applied to the study of circular motion.
%Circular motion is described by the angle of rotation $\theta(t)$, 
%the angular velocity $\omega(t)$ and the angular acceleration $\alpha(t)$.
%The causes of angular acceleration are angular force, which we call \emph{torques} $\mcal{T}$.
%Apart from this change to angular quantities, the principles behind the circular motion 
%are exactly the same as those for linear motion.
%and Newton's second law for rotational motion is $\mcal{T}_{net}=I\alpha$.
%We will 
%The quantity $I$ is called the \emph{moment of inertia} and describes how difficult it is to make an object turn.
%Circular motion is therefore 
%\begin{equation}
% \frac{1}{I}\sum \mathcal{T} = \alpha(t) \ \overset{\omega_i+ \int\!dt }{\longrightarrow} \ \omega(t) \ \overset{\theta_i+ \int\!dt }{\longrightarrow} \ \theta(t).
% \label{fma-eqn}
%\end{equation}


%\begin{equation}
%  U(h) = - \int_0^h  \vec{F}_g \cdot d\vec{y} = - \int_0^h  (- mg\hat{\jmath}) \cdot \hat{\jmath} \;dy    = mgh.
%  \nonumber
%\end{equation}

%
%During a collision between two objects there will be a sudden spike in the contact force between them,
%which can be difficult to measure and quantify.
%It is therefore not possible to use Newton's law $F=ma$ to predict the accelerations that occur during collisions.
%%and the resulting motion of the objects after they collide.
%In order to predict the motion of the objects after the collision we must use a \emph{momentum} calculation.
%An object of mass $m$ moving with velocity $\vec{v}$ has momentum $\vec{p}\equiv m\vec{v}$.
%The principle of conservation of momentum states that {\bf the total amount of momentum before and
%after the collision remains constant}. Thus, if two objects with initial momenta $\vec{p}_{i1}$ and $\vec{p}_{i2}$ collide,
%the total momentum before in the collision must be equal to the total momentum after the collision:
%\[
%	\sum \vec{p}_i = \sum \vec{p}_f 	
%	\qquad
%	\Rightarrow
%	\qquad
%	\vec{p}_{i1} + \vec{p}_{i2} 
%		=
%		\vec{p}_{f1} + \vec{p}_{f2}.
%\]
%Using this equation, it is possible to calculate the final momenta $\vec{p}_{f1}$, $\vec{p}_{f2}$
%of the objects after the collision. 

%
%Another way of solving physics problems is to use the concept of energy.
%Instead of trying to describe the entire motion of the object, 
%we can focus only on the initial parameters and the final parameters.
%The law of conservation of energy states that {\bf the total energy of the system is a constant}.
%Knowing the initial energy of a system, therefore, allows us to infer the final energy.


In the remainder of this document, 
we will learn more about each of the above concepts and ways of thinking. 
Before we begin with the physics material, we must introduce some mathematical
background, which will allow us to better understand the concepts.

