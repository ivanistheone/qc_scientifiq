\documentclass[journal]{IEEEtran}

\title{ \Huge Equation quadratique}
\author{Ivan Savov} 


\usepackage{amsthm}
\usepackage{amsmath}
\usepackage{amssymb}
\usepackage{hyperref}

\def\mcal{\mathcal}
\def\eps{\epsilon}

\newcommand{\comment}[1]{\noindent [\textit{#1}]}





\newcommand{\settexitref}[2]{(\ref{#1}p\pageref{#1})}
\newcommand{\dokutitlelevelone}[1]{\chapter{#1}}
\newcommand{\dokutitleleveltwo}[1]{\section{#1}}
\newcommand{\dokutitleleveltree}[1]{\subsection{#1}}
\newcommand{\dokutitlelevelfour}[1]{\subsubsection{#1}}
\newcommand{\dokutitlelevelfive}[1]{\paragraph{#1}}
\newcommand{\dokufootnote}[1]{\footnote{#1}}
\newcommand{\dokufootmark}[1]{\footnotemark{#1}}
\newcommand{\dokubold}[1]{\textbf{#1}}
\newcommand{\dokuitalic}[1]{\textsl{#1}}
\newcommand{\dokumonospace}[1]{\texttt{#1}}
\newcommand{\dokuunderline}[1]{\underline{#1}}
\newcommand{\dokuoverline}[1]{\sout{#1}}
\newcommand{\dokusupscript}[1]{\textsuperscript{#1}}
\newcommand{\dokusubscript}[1]{$_{#1}$}
\newcommand{\dokuhline}{\line(1,0){400}}
\newcommand{\dokulabel}[1]{\label{#1}}
\newcommand{\dokuitem}{\item}
\newcommand{\dokuquoting}{\textbar}
\newcommand{\dokutabularwidth}{\textwidth}
% added by Ivan Savov
\newcommand{\dokuheadingstyle}[1]{#1}






\usepackage{tikz}
\usetikzlibrary{arrows,shapes,decorations,automata,backgrounds,petri}
\usetikzlibrary{shapes.gates.logic.US}
\usepackage[latin1]{inputenc}
% http://tex.stackexchange.com/questions/13933/drawing-mechanical-systems-in-latex/13952#13952
\usetikzlibrary{calc,patterns,decorations.pathmorphing,decorations.markings}





\begin{document}
\maketitle


\begin{abstract}
We review some basic math notions and the 
quadratic equation.
\end{abstract}





\dokutitlelevelfour{Example 1}

Let's say your teacher doesn't like you and right away on the first day of classes he gives you 
a serious equation and wants you to find \(x\):
\[
   \log_5\left(3 + \sqrt{6\sqrt{x}-7}   \right) = 34+\sin(345)-\Psi(1).
\]
Do you see now what I meant when I said that the teacher doesn't like you?

First note that it doesn't matter what \(\Psi\) is, since \(x\) is on the other 
side of the equation. We can just keep copying \(\Psi(1)\) from line to line
and throw the ball back to the teacher: ``My answer is in terms of your variables dude\ldots{} \dokuitalic{you} go ahead and figure out what the hell \(\Psi\) is,
since you brought it up.''
The same goes with \(\sin(345)\). If you don't have a calculator don't worry about it. 
We will just keep \(\sin(345)\) around and not try to evaluate it.
Besides would you type in \(345\) radians or \(345\) degrees?


\dokubold{Let's just find \(x\) and get it over with!}

On the right side we have the sum of a bunch of terms and no \(x\) in them so we can leave them alone.
On the left hand side, the outer most function is a logarithm base \(5\). 
Cool\ldots{}. no problem look up in the table and find the 
row where we have: \(a^x \leftrightarrow \log_a(x)\).
We are going to apply the inverse on both sides:
\[
   5^{ \log_5\left(3 + \sqrt{6\sqrt{x}-7}   \right) }  = 5^{ 34+\sin(345)-\Psi(1) },
\]
which simplifies to:
\[
  3 + \sqrt{6\sqrt{x}-7} = 5^{ 34+\sin(345)-\Psi(1) }.
\]

From here on it's like if Bruce Lee walked into a place with lots of bad guys. Addition of 3 is undone by subtracting \(3\) on both sides:
\[
  \sqrt{6\sqrt{x}-7} = 5^{ 34+\sin(345)-\Psi(1) } - 3,
\]
and to undo a square root you take the square
\[
  6\sqrt{x}-7 = \left(5^{ 34+\sin(345)-\Psi(1) } - 3\right)^2.
\]
Add \(7\) to both sides
\[
  6\sqrt{x} = \left(5^{ 34+\sin(345)-\Psi(1) } - 3\right)^2+7.
\]
Divide by \(6\):
\[
  \sqrt{x} = \frac{1}{6}\left(\left(5^{ 34+\sin(345)-\Psi(1) } - 3\right)^2+7\right),
\]
and then we square again plus some simplifications to get:
\begin{align*}
  x 
 &= \left[\frac{1}{6}\left(\left(5^{ 34+\sin(345)-\Psi(1) } - 3\right)^2+7\right) \right]^2  \\
 &= \frac{1}{36}\left[\left(5^{ 34+\sin(345)-\Psi(1) } - 3\right)^2+7\right]^2.
\end{align*}

Do you see what I am doing?
Next time a function stands in your way, hit it with its inverse!


\dokutitlelevelfour{Discussion}

The recipe I have outlined above is not universal.
Sometimes \(x\) isn't alone on one side. Sometimes \(x\) appears in several places in the same equation
so just throwing complications from one side to the other won't help you.
You need other techniques for solving equations like that.

The bad news is that there is no general formula for solving complicated formulas.
The good news is that the above technique of digging towards \(x\) is sufficient for 80\% 
of what you are going to be doing.
You can get another 10\% if you learn how to solve the \hyperref[470dad9cac00a9924004d164cb491a31]{quadratic{\textbackslash}\_equation}:
\[
  ax^2 +bx + c = 0.
\]

Solving third order equations \(ax^3+bx^2+cx+d=0\) with 
pen and paper is still possible, but at this point you really
might as well start using a computer. 

There are all kinds of other equations which you can learn how
to solve: equations with multiple unknowns, equations with
logarithms, equations with exponentials, and equations
with trigonometric functions.
The essential principle of \dokuitalic{digging}
towards the unknown will get you pretty far though, so be sure
to practice it.





\dokutitleleveltwo{Basic algebra rules}
\label{c3e38dc40d7a0015b4d8b65d875bd50b}%% basic_algebra_rules

For \(a,b,c,d\in\mathbb{R}\) holds:


\begin{enumerate}\dokuitem  Distributive property: \(a(b+c)=ab+ac\).
\dokuitem  Associative property: \(a(bc)=b(ac)=c(ab)\).
\dokuitem  Commutative property: \(a+b=b+a\), \(ab=ba\).
\end{enumerate}

Combining the distributive and the associative property we can get this: \\ 
\((a+b)(c+d)=a(c+d) + b(c+d) = ac+ad+bc+bd\).


\dokutitlelevelfive{Example}
This is useful when you are dealing with polynomials:
\[
  (x+3)(x+2)=x(x+2) + 3(x+2)= x^2 +x2 +3x + 6.
\]
Now for polynomials, we can use the commutative property
on the second term \(x2=2x\), and then combine the
two \(x\) terms into one to obtain:
\[
  (x+3)(x+2)= x^2 + 5x + 6.
\]


This kind of calculation happens so often that it is
good idea to see it in more abstract form
\[
  (x+a)(x+b) = x^2 + (a+b)x + ab.
\]
My high school teacher used to repeat this to us all
the time ``a binomial times a binomial gives a trinomial''.


\dokutitleleveltree{Factoring}
\label{bd72fce91eca7fd86b64fe98d212427e}%% factoring

Factoring means to take out some common part in a
complicated expression to make it more compact.
Given an expression, we can ask what the 
factors of each term are:
\[
  6x^2y + 15x =   (3)(2)(x)(x)y + (5)(3)x.
\]
We see that \(x\) and \(3\) appear in both terms.
This means we can \dokuitalic{factor them out} to the front
like this:
\[
 6x^2y + 15x =  3x(2xy+5).
\]
Now is this simpler? 
On the left there are \(8\) characters and on the 
right there are \(7\) characters so we are saving space.
More significant savings can occur in expressions like this:
\[
  2x^2y + 2x + 4x = 2x(xy+1+2) =  2x(xy+3).
\]


\dokutitlelevelfive{Example}

You are asked to solve for \(t\) in 
\[
 7(3 + 4t) = 11(6t - 4).
\]
As you can see, \(t\) is on both sides. We don't like that.
We have to bring all the \(t\) terms to one side and all the
constant terms on the other.
First let's expand the two brackets:
\[
 21 + \tikz[remember picture] \node[fill=blue!20] (n2) {$28t$};  = 66t - 44.
\]
Now we move things around to get:
\[
 21 + 44 = 66t -  \tikz[remember picture] \node[fill=blue!20] (n2bis) {$28t$};  .
\]
If we factor out \(t\) on the right hand side we get:
\[
 21 + 44 = (66 - 28)t,
\]
so \(t = \frac{21 + 44}{66 - 28} = \frac{65}{38}\).

\begin{tikzpicture}[remember picture,overlay] 
	\draw[blue!20,->] (n2) -- (n2bis);
\end{tikzpicture}

\dokutitleleveltree{Quadratic factoring}
\label{231e44a9b24528f89d83cbc42d87e094}%% quadratic_factoring

When dealing with a quadratic functions, it is often
useful to rewrite it as a product of two terms.
If I give you the quadratic expression
\[
  x^2+5x+6,
\]
you can't really tell what its properties are.
What are its roots? Where is it positive and where is it negative?

We can factor this expression, that is express it as a product of two things \((x-?)(x-?)\)
as follows:
\[
  x^2+5x+6 = (x+2)(x+3),
\]
and you see immediately that \(x=-2\) and \(x=-3\) are the solutions (roots).
You can also see that for \(x>-2\) the expression is positive, since 
both terms will be positive. For \(x< -3\) both terms will be negative, but
negative times negative gives positive, so the expression will be positive again.
For values of \(x\) such that \(-3<x< -2\), the first term will be negative, and
the second positive so the overall expression will be negative.

For some simple quadratics you can \dokuitalic{guess} what the factors are.
For more complicated ones, you need the \hyperref[470dad9cac00a9924004d164cb491a31]{quadratic formula}.

Quadratic functions that never cross the \(x\)-axis cannot be factored.


\dokutitleleveltree{Completing the square}
\label{b94fc77efd4263b48096cc83d6dee58b}%% completing_the_square

Any quadratic formula can be written in the form \(A(x-h)^2+k\).
This is because all quadratic function with the same quadratic coefficient 
are essentially shifted version of each other. 
By \dokuitalic{completing the square} we are making these shifts explicit. 
The value of \(h\) is how much the function is shifted to the \dokuitalic{right} and
the value \(k\) is the vertical shift.

Let's try to find the values \(A,k,h\) for our previous formula
\[
  x^2+5x+6  =   A(x-h)^2+k = A(x^2-2hx + h^2) + k = Ax^2 - 2Ahx + Ah^2 + k.
\]

By focussing on quadratic terms on both sides of the equation
we see that \(A=1\), so we have
\[
  x^2+\underline{5x}+6 =  x^2  \underline{-2hx} + h^2 + k.
\]
Next we look at the terms multiplying \(x\), and we see that \(h=-2.5\),
so we obtain
\[
  x^2+5x+\underline{6} =  x^2+ 2(2.5)x + \underline{(2.5)^2 + k}.
\]
Finally we pick value of \(k\) which would make the constant terms match
\[
 k = 6 - (2.5)^2 = 6 - \left(\frac{5}{2}\right)^2 = 6 \times \frac{4}{4} - \frac{25}{4} = \frac{24 - 25}{4} = \frac{-1}{4}.
\]
This is how we complete the square, to obtain:
\[
  x^2+5x+6  = (x+2.5)^2 - \frac{1}{4},
\]
which tells us that our function was just the basic \(x^2\), shifted 2.5 units to the left, and \(1/4\) units downwards.
This would be really useful information if you ever had to draw this function, since it is easy to plot the basic \(x^2\) graph, and then shift
it appropriately.




\dokutitleleveltwo{Solving quadratic equations}
\label{194a7a851bde33e69ef4857d4aec9049}%% solving_quadratic_equations

If you are asked to find \(x\) in the equation 
\[
  x^2 = 45x + 23,
\]
what would you do?

In math this happens a \dokuitalic{whole} lot, so people came up with a 
formula for solving it.

To apply the formula you need to put the equation into the
standard form
\[
  ax^2 + bx + c = 0,
\]
where we have moved all the numbers and \(x\)'s to one side and left 
just \(0\) on the other side.

For our numerical example above this would mean that we subtract 
\(45x+23\) on both sides of the equation to obtain
\[
  x^2 - 45x - 23 = 0.
\]

What are the values of \(x\) that satisfy this formula?


\dokutitlelevelfour{Claim}

The solutions to the equation 
\[
  ax^2 + bx + c = 0,
\]
are
\[
   x_1 = \frac{-b + \sqrt{b^2-4ac} }{2a}, \ \ \text{  and  }
 \ \ x_2 = \frac{-b - \sqrt{b^2-4ac} }{2a}.
\]


Using the quadratic formula, solving the above complicated
equation \(x^2 - 45x - 23 = 0\) becomes a mechanics task of plugging
in the appropriate numbers into the formula:
\[
  x_1 = \frac{45 + \sqrt{45^2-4(1)(-23)} }{2} = 45.5054\ldots,
\]
\[
  x_2 = \frac{45 - \sqrt{45^2-4(1)(-23)} }{2} = -0.5054\ldots
\]
Answers obtained by using \href{http://www.google.ca/search?hl=en&client=safari&rls=en&q=%28++45+%2B+sqrt%2845%5E2-4%281%29%28-23%29%29++++%29%2F2&aq=f&aqi=&aql=&oq=&gs_rfai=}{google}.


\dokutitlelevelfour{Proof of claim}

This is an important proof. You should know how to derive the quadratic
formula in case your younger brother asks you one day 
to derive the formula from first principles.
To understand the proof, all you need to know is
how to ``complete the square'',
which we just saw, so you can't say that you are not prepared.


We start with
\[
  ax^2 + bx + c = 0.
\]
As a first step we move \(c\) to the other side
\[
  ax^2 + bx  = -c,
\]
and then we divide by \(a\) on both sides
\[
  x^2 + \frac{b}{a}x  = -\frac{c}{a}.
\]

Now we \dokuitalic{complete the square} on the left hand side, 
which is to say we ask the question: what are the values of \(h\) and \(k\)
for this equation to hold?
\[
  (x+h)^2 + k = x^2 + \frac{b}{a}x = -\frac{c}{a}.
\]
The answer is \(h = \frac{b}{2a}\).
It can be a bit confusing to figure out what \(k\) is, so let's
work it out step by step.
Just remember that \dokubold{to complete the square you 
have to use half of the linear term inside the bracket.}

So what do we have so far:
\[
  \left(x + \frac{b}{2a} \right)^2
= \left(x + \frac{b}{2a} \right)\!\!\left(x + \frac{b}{2a} \right)
= x^2 + \frac{b}{2a}x + x\frac{b}{2a} + \frac{b^2}{4a^2}  = x^2 + \frac{b}{a}x + \frac{b^2}{4a^2},
\]
which means that 
\[
  \left(x + \frac{b}{2a} \right)^2 - \frac{b^2}{4a^2}  = x^2 + \frac{b}{a}x.
\]

Remember where we left off in the proof?
\[
  x^2 + \frac{b}{a}x  = -\frac{c}{a}
\]
We replace the left hand side by the completed square expression:
\[
   \left(x + \frac{b}{2a} \right)^2 - \frac{b^2}{4a^2}  = -\frac{c}{a}.
\]
From here on it is standard stuff from \hyperref[de8b0e5156f4d2cb51a8bfd6e08459a0]{solving{\textbackslash}\_equations}.
We put all the constants on the same side
\[
   \left(x + \frac{b}{2a} \right)^2  = -\frac{c}{a} +  \frac{b^2}{4a^2}, 
\]
Since \(x^2\) maps both positive and negative numbers to the same value,
there will be two solutions:
\[
   x + \frac{b}{2a}  = \pm \sqrt{ -\frac{c}{a} +  \frac{b^2}{4a^2}  }.
\]
Let's take this moment to cleanup the mess on the right hand side.
\[
  \sqrt{ -\frac{c}{a} +  \frac{b^2}{4a^2}  } 
= \sqrt{ -\frac{(4a)c}{(4a)a} +  \frac{b^2}{4a^2}  }
= \sqrt{ \frac{- 4ac + b^2}{4a^2}  }
= \frac{\sqrt{b^2 -4ac}   }{ 2a   }.
\]

Thus we have:
\[
   x + \frac{b}{2a}  = \pm \frac{\sqrt{b^2 -4ac}   }{ 2a   },
\]
which is just one step away from the final answer
\[
   x  = \frac{-b}{2a}  \pm \frac{\sqrt{b^2 -4ac}   }{ 2a   } = \frac{-b \pm \sqrt{b^2 -4ac}   }{ 2a   },
\]
which completes the proof.


\dokutitlelevelfour{Alternative proof of claim}

We didn't need to necessarily show the derivation of the formula
as we did.  The claim was that \(x_1\) and \(x_2\) are solutions,
so to prove the claim we could simply plug in \(x_1\)
and \(x_2\) into the quadratic equation and verify that we get zero.
\dokubold{Verify on your own.}


\dokutitleleveltree{Applications}
\label{b5fba9ff24d0045d1377a05a46b32f68}%% applications

\dokutitlelevelfour{The Golden Ratio}

The golden ratio, usually denoted \(\phi=1.6180339\ldots\)
is a very important  proportion in geometry, art, aesthetics, biology and mysticism. 
It comes about from the solution to the equation
\[
  x^2 +x -1 = 0.
\]

Using the quadratic formula we get the two solutions
which are
\[
   x_1 = \frac{-1+\sqrt{5}}{2} = \frac{1}{\phi}, \qquad x_2 = \frac{-1-\sqrt{5}}{2} = - \phi.
\]

You can \hyperref[d7bbf6d9d4bc796ac89630339c203241]{read more about the golden ratio} to learn
about the verious contexts in which it appears.




\dokutitleleveltwo{Exponents}
\label{2f711bc1acfc4fcc18827d8a2655a05f}%% exponents

We often have to multiply number together the same number many times
in math so we use the notation
\[
  x^n  = \underbrace{xxx \cdots xx}_{n \text{ times} }
\]
to denote some number \(x\) multiplied together \(n\) times.


\dokutitleleveltree{Definitions}
\label{9fdc1f6b239f0e86ec8651552f2b0683}%% definitions

\begin{itemize}
\dokuitem  \(b^x\): \(b\) exponent \(x\):
\begin{itemize}
\dokuitem  \(b\): is called the \dokuitalic{base}.
\dokuitem  \(x\): is the exponent.
\end{itemize}

\dokuitem  \(\exp(x)=e^x\): the ``natural'' exponential function base \(e\)  Euler's number.
\dokuitem  \(2^x\): exponential function base \(2\), very important in computer science.
\dokuitem  \(b^0=1\): by definitions zero times something gives \(1\).
\end{itemize}

\dokutitleleveltree{Formulas}
\label{51d24e1edefe34e683025dbba5c6eed6}%% formulas

The number \(e=2.7182818\ldots\) is a special base that has lots of applications.
We call that the \dokuitalic{natural} base.

Another special base is \(10\) because we use the decimal
system for our numbers.
The \(\log_{10}\) of a number corresponds to 
how many digits there are in the number. 


\dokutitlelevelfive{Property 1}

\[
   b^m b^n = b^{m+n},
\]
which follows from first principles:
Exponentiation means multiplying together the base many times.
So \(b^m\) means you multiply \(b\) times itself \(m\) times. So if you
count the total number of \(b\)s on the left side you will see that
there is a total of \(m+n\) of them.


\dokutitlelevelfive{Property 2}
\[
   b^{-n} = \frac{1}{b^n},
\]


\dokutitlelevelfive{Property 3}
\[
   \frac{b^m}{b^n} = b^{m-n},
\]


\dokutitlelevelfive{Property 4}
\[
   ({b^m})^n = b^{mn},
\]


\dokutitlelevelfive{Property 5}
\[
   (ab)^n = a^n b^n,
\]


\dokutitlelevelfive{Property 5.2}
\[
   \left(\frac{a}{b}\right)^n = \frac{a^n}{b^n},
\]


\dokutitlelevelfive{Property 6}
\[
  b^{1/n} = \sqrt[n]{b},
\]
in particular \(b^{1/2}=\sqrt{b}\).

\[
  \sqrt[n]{ab} = \sqrt[n]{a}\sqrt[n]{b},
\]
\[
  \sqrt[n]{\left(\frac{a}{b}\right)} = \frac{\sqrt[n]{a} }{ \sqrt[n]{b} },
\]

If \(n\) is odd
\[
  \left( \sqrt[n]{b} \right)^n = \sqrt[n]{ b^n } = b.
\]

If \(n\) is even and \(b\) is positive
\[
  \left( \sqrt[n]{b} \right)^n = \sqrt[n]{ b^n } = |b|,
\]
but if \(b\) is negative then we can't evaluate \(\sqrt[n]{b}\)
(there is no real number which multiplied times itself gives a negative number)
so only the second part of the ``cancellation'' holds
\[
   \sqrt[n]{ b^n } = |b|,
\]
where \(|b|\) is the absolute value.




\dokutitleleveltwo{Logarithms}
\label{c95a6da6fb331984f198feec0b0bb1a5}%% logarithms

The word ``logarithms'' makes most people think about some mythical
mathematical beast. Probably many headed and extremely difficult to
understand.

None such thing!  Logarithms are simple. It will take you 
at most a couple of pages to get used to manipulating them. 
Logarithms are used all over the place.
The strength of your sound system is measured in decibels \([dB]\), 
which is because your ear is sensitive only to exponential differences
in sound.
Logarithms allow us to compare very large numbers and very 
small numbers on the same scale. If we were measuring sound
in linear units instead of logarithmic then your sound system
volume control would have to go from \(1\) to \(1048576\). 
Indeed, if the volume notches on your sound system go from 1 to 20,
then each notch probably doubles the power 
instead of increasing it by a fixed amount.

Let's get started.


\dokutitleleveltree{Definitions}
\label{9fdc1f6b239f0e86ec8651552f2b0683}%% definitions

You are probably familiar with these concepts already:


\begin{itemize}\dokuitem  \(b^x\): the exponential function base \(b\).
\dokuitem  \(\exp(x)=e^x\): the ``natural'' exponential function base \(e\)  Euler's number.
\dokuitem  \(2^x\): exponential function base \(2\), very important in computer science.
\dokuitem  \(f(x)\): some function.
\dokuitem  \(f^{-1}(x)\): the inverse of the function \(f(x)\). It is defined in terms of \(f(x)\)  such that the following holds \(f^{-1}(f(x))=x\) i.e.,  if you apply \(f\) to some number and get the output \(y\), and then you feed \(y\) into \(f^{-1}\) the output of \(f^{-1}\) will be \(x\).
\end{itemize}

In this section we will play with the following new concepts:


\begin{itemize}\dokuitem  \(\log_b(x)\): logarithm of \(x\) base \(b\). This is the inverse function of \(b^x\).
\dokuitem  \(\ln(x)\); the ``natural'' logarithm base \(e\). This is the inverse of \(e^x\).
\dokuitem  \(\log_2(x)\): logarithms base \(2\). The answer to the question, how many bits of computer memory does it take to store a large number \(N\)  is \(\log_2(N)\) bits.
\end{itemize}

\dokutitleleveltree{Formulas}
\label{51d24e1edefe34e683025dbba5c6eed6}%% formulas

The main thing to realize is that \(\log\)s don't really exist on their own.
They are defined as the inverses of the corresponding exponential function.
The following statements are equivalent:
\[
  \log_b(x)=m \ \ \ \ \  \Leftrightarrow \ \ \ \ \ b^m=x.
\]

For logarithms with base \(e\) one writes \(\ln(x)\) because
\(e\) is a special base that has lots of applications.
This is a subject for another lesson, but I wanted also
to mention that \(e^x\) is actually related to \(\sin(x)\) and \(\cos(x)\).

Another special base is \(10\) because we use the decimal
system for our numbers. \(\log_{10}(x)\) tells you roughly
the size of the number \(x\): how many digits it has. 
When a system pawn with a high paying financial sector job
boast about his or her ``six-figure'' salary, they are really talking
about the \(\log\) of how much money they make.
What wankers! 

Let me show you the two-three equations that you need to know
for dealing with logs.


\dokutitlelevelfive{Property 1}

\[
  \log(x)+\log(y)=\log(xy),
\]
from which we can derive two other useful ones:
\[
  \log(x^k)=k\log(x),
\]
and 
\[ 
  \log(x)-\log(y)=\log\left(\frac{x}{y}\right).
\]

\dokuitalic{Proof:} For all three equations above we have to show
that the expression on the left is equal to the expression
on the right.

We have only been acquainted with logarithms for a very
short time. We don't know each other that well. 
In fact, the only thing we know about \(\log\)s is the 
inverse relationship with the exponential function.
So the only way to prove this property is to use this relationship.

The following statement is true for any base \(b\):
\[
   b^m b^n = b^{m+n},
\]
which follows from first principles. 
Exponentiation means multiplying together the base many times.
So \(b^m\) means you multiply \(b\) times itself \(m\) times. So if you
count the total number of \(b\)s on the left side you will see that
there is a total of \(m+n\) of them.

If you define some new variables \(x\) and \(y\) such that
\(b^m=x\) and \(b^n=y\) then the above equation will read 
\[
  xy = b^{m+n},
\]
if you take the logarithm of both sides you get
\[
  \log_b(xy) = \log_b\left(  b^{m+n} \right) = m + n = \log_b(x) + \log_b(y).
\]
In the last step we used the definition of log again which
states that \(b^m=x \Leftrightarrow m=\log_b(x)\) and \(b^n=y \Leftrightarrow n=\log_b(y)\).


Next we discuss another set some more rules about changing from
one base to another. 
Is there some relation between \(\log_{10}(S)\) and \(\log_2(S)\)?


\dokutitlelevelfive{Property 2}

\[
  \log_{B}(x) =  \frac{\log_b(x)}{\log_b(B)}
\]




This means that:
\[
 \log_{10}(S) 
 =\frac{\log_{10}(S)}{1}
 =\frac{\log_{10}(S)}{\log_{10}(10)} 
 = \frac{\log_{2}(S)}{\log_{2}(10)}=\frac{\ln(S)}{\ln(10)}.
\]

This property is very useful in case when you want to compute 
\(\log_{10}\), but your calculator only gives you \(\ln\) then you simply compute the 
two numbers on the right hand side and divide them.


\dokutitleleveltree{Discussion}
\label{bd8bc36eb41bc90c585ae7e902e9e284}%% discussion

\dokutitlelevelfive{Applications of logarithms}

In many sciences people often interchange multiplication and addition 
by using a logarithm transform
\[
  \log_3(f(x)) + log_3(g(x)) = z,
\]
which by the properties of logarithms (see definition) can be written as  
\[
  \log_3(f(x)g(x)) = z,
\]
which in turn means \(f(x)g(x)=3^z\).

This is useful in probability theory since if you have some probability distribution \(p(x,y)\) that is independent then \(p(x,y)=p_X(x)p_Y(y)\). If you take the logarithm of this expression (the concept of log likelihood is the basis of \dokuitalic{machine learning}) then you
you get
\[
  \log(p_X(x)p_Y(y)) = \log(p_X(x)) + \log(p_Y(y)),
\]
so very complicated product expressions can be evaluated as sums.
Human and computers are much better at adding numbers 
than multiplying them, so we use this trick all the time.




\section{Strudent loan interest rate}

We need the exponental function here...






%\bibliographystyle{alpha}
%\bibliography{interferenceChannel}

\end{document}
