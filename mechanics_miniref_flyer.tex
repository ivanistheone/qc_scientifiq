\documentclass[letterpaper,9pt,journal]{IEEEtran}


\def\fourrr{four }
%\documentclass[twocolumn,8pt]{extarticle}
\title{ \Huge MECHANICS explained in \fourrr pages}
\author{Excerpt from the \emph{MATH and PHYSICS Minireference} by Ivan Savov} 


%\usepackage[papersize={5.5in,8.5in},verbose,bmargin=0.9cm,rmargin=0.7cm,lmargin=0.7cm,tmargin=0.9cm,headsep=0.3cm,footskip=0.5cm]{geometry}

\usepackage[bmargin=1cm,rmargin=0.9cm,lmargin=0.9cm,tmargin=1cm,headsep=0.2cm,footskip=0.5cm]{geometry}

\usepackage[english]{babel} %% Use your own babel language
\date{}


\usepackage{amsthm}
\usepackage{amsmath}
\usepackage{amssymb}
\usepackage{hyperref}

\def\mcal{\mathcal}
\def\eps{\epsilon}

\newcommand{\comment}[1]{\noindent [\textit{#1}]}



\usepackage{wrapfig}



%\usepackage{standalone}

%\usepackage{enumitem}
%\setitemize{itemsep=-0.02in}   % LISTS WERE TOO AIRY



\newcommand{\settexitref}[2]{(\ref{#1}p\pageref{#1})}
\newcommand{\dokutitlelevelone}[1]{\chapter{#1}}
\newcommand{\dokutitleleveltwo}[1]{\section{#1}}
\newcommand{\dokutitleleveltree}[1]{\subsection{#1}}
\newcommand{\dokutitlelevelfour}[1]{\subsubsection{#1}}
\newcommand{\dokutitlelevelfive}[1]{\paragraph{#1}}
\newcommand{\dokufootnote}[1]{\footnote{#1}}
\newcommand{\dokufootmark}[1]{\footnotemark{#1}}
\newcommand{\dokubold}[1]{\textbf{#1}}
\newcommand{\dokuitalic}[1]{\textsl{#1}}
\newcommand{\dokumonospace}[1]{\texttt{#1}}
\newcommand{\dokuunderline}[1]{\underline{#1}}
\newcommand{\dokuoverline}[1]{\sout{#1}}
\newcommand{\dokusupscript}[1]{\textsuperscript{#1}}
\newcommand{\dokusubscript}[1]{$_{#1}$}
\newcommand{\dokuhline}{\line(1,0){400}}
\newcommand{\dokulabel}[1]{\label{#1}}
\newcommand{\dokuitem}{\item}
\newcommand{\dokuquoting}{\textbar}
\newcommand{\dokutabularwidth}{\textwidth}
% added by Ivan Savov
\newcommand{\dokuheadingstyle}[1]{#1}


\newcommand{\be}{\begin{equation}}
\newcommand{\ee}{\end{equation}}




\usepackage{tikz}
\usetikzlibrary{arrows,shapes,decorations,automata,backgrounds,petri}
\usetikzlibrary{shapes.gates.logic.US}
\usepackage[latin1]{inputenc}
% http://tex.stackexchange.com/questions/13933/drawing-mechanical-systems-in-latex/13952#13952
\usetikzlibrary{calc,patterns,decorations.pathmorphing,decorations.markings}





\begin{document}
\maketitle

%\vspace*{-1cm}

%

%\twocolumn[
%  \begin{@twocolumnfalse}
%\begin{center}
%\fontsize{27}{16}\selectfont
%%EZ \ MECHANICS \ TUTORIALS
%MECHANICS \ \  in \ \  FOUR PAGES
%\normalsize 
%%\vspace{2mm}
%%by Ivan Savov
%\vspace{6mm}
%\end{center}
%	%    \maketitle
%	%    \begin{abstract}
%	%      ...
%	%    \end{abstract}
%  \end{@twocolumnfalse}
%  ]
  

\begin{abstract}
%This document is a self contained tutorial on Mechanics.
Mechanics is the precise study of the motion of objects, the forces acting
on them and more abstract concepts such as momentum and energy.
You probably have an intuitive understanding of these concepts already,
but in the next \fourrr pages you will learn how to use precise mathematical
equations to support your intuition. Perhaps more importantly, we will also learn
where the equations come from.
%There will be parties to go to and beers to drink later on in the semester, 
%so it is best to learn about all the important concepts of physics right now 
%and have time to chill later on in the semester. 
%%
%If you understand everything that is covered in the next \fourrr pages,
%I can guarantee you that you will be able to pass the final exam.
\end{abstract}



%\dokutitlelevelone{Mechanics}
%\label{05f8e33b48d5f8430980e3d1f38e5116}%% mechanics

%Mechanics is the precise study of moving objects, forces and energy.
%You already have an intuitive understanding of these concepts,
%but in this chapter I will teach you how to use precise mathematical 
%models which will support your intuition.
%
%Mechanics is the part of physics that is most well understood.
%Ever since Newton figured out the whole \(F=ma\) thing and 
%the law of gravitation, people have \dokuitalic{used} mechanics in order
%to achieve great technological feats. 

%There will be math, yes, but nothing too complicated.
%In fact, the hardest type of equation you will have to solve is a quadratic equation, so don't worry too much about that.
%The upshot of understanding the math, is that you will 
%be able to calculate and predict phenomena in the world around you
%simply by plugging numbers into the right equation.

\section{Introduction}

To solve a physics problem is to obtain the \emph{equation of motion} $x(t)$, 
which describes position of the object as a function of time.
%
Once you know $x(t)$, you can answer many of the question pertaining to the motion of the object.
To find the initial position $x_i$ of the object, you simply plug $t=0$ into the equation of motion $x_i = x(0)$.
To find the time(s) when the object reaches a distance of 20[m] from the origin, we simply solve for $t$ in $x(t)=20$[m].
Many of the problems on you final exam in physics will be of this form, so it is really important that you know
how to find the equation of motion for any object.

%\paragraph{{\bf Forces}}
The first step in finding $x(t)$ is to calculate all the \emph{forces} that act on the object.
Forces are the \emph{cause} of motion, so if you want to understand motion you need to understand forces. 
Newton's second law $F=ma$ states that  the amount force acting on an object produces an \emph{acceleration}
inversely proportional to the mass of the object. 
Once you have the acceleration, you can compute $x(t)$ using two simple calculus steps.
For now, though, we want to focus on the causes of motion: the forces acting on the object.
There are many kinds of forces: the weight of an object $\vec{W}$ is a type force, 
the force of friction $\vec{F}_f$ is another type another, the tension in a rope $\vec{T}$ yet another type of force
and there are many others.
Note the little arrow on top of each force, which is there to remind you that forces are \emph{vector} quantities.
Unlike regular numbers, forces act in a particular direction, so it is possible that the effects of 
one force are counteracting another force. For example the force of the weight of a flower pot 
is exactly counter-acted by the tension in the rope on which it is suspended, thus,
while there are two forces that may be acting on the pot, there is no \emph{net force} acting on it.
Since there is no net force to cause motion and since the pot wasn't moving to begin with, 
it will just sit there motionless despite the fact that there are forces acting on it!
The first step when analyzing a physics problem will be calculation of the \emph{net force}
acting on the object, which is the sum of all the forces acting on the object: $\vec{F}_{net} \equiv \sum \vec{F}$.
Once we have found the net force we can use $\vec{F}_{net}=m\vec{a}$ to find the resulting acceleration.

It turns out that once you know the acceleration of an object $a(t)$, 
you can easily find its velocity $v(t)$ function and once you know the velocity 
function you can find the position function $x(t)$, which is what we wanted to find in the first place.
The acceleration is the change in the velocity of the object, thus if you know
that the object stated with an initial velocity of $v_i \equiv v(0)$,
and you want to find the velocity at later time $t=\tau$, 
you have to add up all the acceleration that the object felt during this time $v(\tau)=v_i+\int_0^\tau a(t)\;dt$.
The symbol $\int \cdot \;dt$ is called an \emph{integral} and is fancy way of finding the total
of some quantity over time. In order to find $x(t)$, we perform a second integration step
in which we add up all the changes in the position (velocity) in order to find the final position $x(t)$.
%
We can summarize the entire process which is used to find $x(t)$ in the following equation:
\begin{equation}
 \frac{1}{m}\sum \vec{F} = a(t) \ \overset{v_i+ \int\!dt }{\longrightarrow} \ v(t) \ \overset{x_i+ \int\!dt }{\longrightarrow} \ x(t).
 \label{fma-eqn}
\end{equation}
%
The right hand side of the equation is called a \emph{dynamics} problem and involves 
the calculation of the \emph{net force} $\vec{F}_{net} \equiv \sum \vec{F}$.
The right hand side in the above equation is called \emph{kinematics} and focusses on
the use of integration in order to find $v(t)$ from $a(t)$ and $x(t)$ from $v(t)$.
Don't worry about this integration business; 
it is quite simple and we will cover everything you need to know about it in the next section.  


Newton's second law can also be applied to the study of circular motion.
Circular motion is described by the angle of rotation $\theta(t)$, 
the angular velocity $\omega(t)$ and the angular acceleration $\alpha(t)$.
The causes of angular acceleration are angular force, which we call \emph{torques} $\mcal{T}$.
Apart from this change to angular quantities, the principles behind the circular motion 
are exactly the same as those for linear motion.
%and Newton's second law for rotational motion is $\mcal{T}_{net}=I\alpha$.
%We will 
%The quantity $I$ is called the \emph{moment of inertia} and describes how difficult it is to make an object turn.
%Circular motion is therefore 
%\begin{equation}
% \frac{1}{I}\sum \mathcal{T} = \alpha(t) \ \overset{\omega_i+ \int\!dt }{\longrightarrow} \ \omega(t) \ \overset{\theta_i+ \int\!dt }{\longrightarrow} \ \theta(t).
% \label{fma-eqn}
%\end{equation}


%\begin{equation}
%  U(h) = - \int_0^h  \vec{F}_g \cdot d\vec{y} = - \int_0^h  (- mg\hat{\jmath}) \cdot \hat{\jmath} \;dy    = mgh.
%  \nonumber
%\end{equation}


During a collision between two objects there will be a sudden spike in the contact force between them,
which can be difficult to measure and quantify.
It is therefore not possible to use Newton's law $F=ma$ to predict the accelerations that occur during collisions.
%and the resulting motion of the objects after they collide.
In order to predict the motion of the objects after the collision we must use a \emph{momentum} calculation.
An object of mass $m$ moving with velocity $\vec{v}$ has momentum $\vec{p}\equiv m\vec{v}$.
The principle of conservation of momentum states that {\bf the total amount of momentum before and
after the collision is remains constant}. Thus, if two objects with initial momenta $\vec{p}_{i1}$ and $\vec{p}_{i2}$ collide,
the total momentum before in the collision must be equal to the total momentum after the collision:
\[
	\sum \vec{p}_i = \sum \vec{p}_f 	
	\qquad
	\Rightarrow
	\qquad
	\vec{p}_{i1} + \vec{p}_{i2} 
		=
		\vec{p}_{f1} + \vec{p}_{f2}.
\]
Using this equation, it is possible to calculate the final momenta $\vec{p}_{f1}$, $\vec{p}_{f2}$
of the objects after the collision. 


Another way of solving physics problems is to use the the concept of energy.
Instead of trying to describe the entire motion of the object, 
we can focus only on the initial parameters and the final parameters.
The law of conservation of energy states that {\bf the total energy of the system is a constant}.
Knowing the initial energy of a system, therefore, allows us to infer the final energy.


In the remainder of this document, 
we will learn more about each of the above concepts and ways of thinking. 
Before we begin with the physics material, we must introduce some mathematical
background, which will allow us to better understand the concepts.


\section{Preliminaries}

In order to understand the equations of physics you need to be familiar with
vector calculations and how to calculate some basic integrals. 
We will introduce these concepts in the next two subsections.

\subsection{Vectors}

Forces, velocities, and accelerations are vector quantities.
A vector quantity $\vec{v}$ can be expressed in terms of its \dokuitalic{components}
or in terms of its length and direction.
\begin{itemize}
%\dokuitem $\vec{v}=(v_x,v_y)$: A vector.
\dokuitem  \(x\)-axis:  The \(x\)-axis is the horizontal the coordinate system.
\dokuitem  \(y\)-axis: The \(y\)-axis is an axis \dokuitalic{perpendicular} to the \(x\)-axis.
\dokuitem  \(v_x\): the \dokuitalic{component} of \(\vec{v}\) along the \(x\) axis.
\dokuitem  \(v_y\): the \dokuitalic{component} of \(\vec{v}\) along the \(y\) axis.
\dokuitem  \(\hat{\imath}\equiv(1,0),\hat{\jmath}\equiv(0,1)\): Unit vectors in the \(x\) and \(y\) directions. 
\dokuitem  $\|\vec{v}\|$: The length of the vector $\vec{v}$.
\dokuitem	 $\theta$: The angle that $\vec{v}$ makes with the $x$ axis.
%\end{itemize}
%Vectors are expressed with respect to a \dokuitalic{coordinate system}:
%\begin{itemize}
%Any vector can be written as \(\vec{v}=v_x\hat{\imath}+v_y\hat{\jmath}\) or as \(\vec{v}=(v_x,v_y)\).
\end{itemize}

Given the $xy$ coordinate system, we can denote a vector in three equivalent ways:
\[
  \vec{v}  
  \ \equiv \ (v_x,v_y)
  \ \equiv \  v_x\hat{\imath} + v_y\hat{\jmath} 
  %\equiv v_x\hat{x} + v_y\hat{y} \equiv  (v_x,v_y)
  \ \equiv \ 
  \|\vec{v}\| \angle \theta.
\]
%where $\|\vec{v}\|$ denotes the length of the vector $\vec{v}$ and
%$\theta$ is the angle that it makes with the $x$ axis.

Given a vector expressed as a length and direction $\|\vec{v}\| \angle \theta$,
we calculate its components as follows:
\[
  v_x = \|\vec{v}\| \cos\theta, \qquad \text{and} \qquad   v_y = \|\vec{v}\|\sin\theta.
\]
Alternately, a vector expressed in component notation $\vec{v}=(v_x,v_y)$ 
is converted to the length-and-direction form as follows:
\[
 \|\vec{v}\|  = \sqrt{ v_x^2 + v_y^2 }, \qquad \text{and} \qquad \theta = \tan^{-1}\!\left( \frac{ F_{y} }{ F_{x} } \right).
\]
It is important to be able to be able to convert between these two forms;
the component notation is useful for calculations whereas the length-and-direction
form will provide you with a geometric intuition.



The \emph{dot product} between two vectors $\vec{v}$ and $\vec{w}$ 
can be  computed in two different ways:
\[
 \vec{v}\cdot\vec{w} = v_xw_y + v_yw_y = \|\vec{v}\| \|\vec{w}\| \cos\phi,
\]
where $\phi$ is the angle between the vectors $\vec{v}$ and $\vec{w}$.
The dot product calculates how similar the two vectors are.
For example, we have $\hat{\imath} \cdot \hat{\jmath} =0$,
since the vectors $\hat{\imath}$ and $\hat{\jmath}$ are orthogonal -- they point in completely different directions.
%


%Calculus involves $f(t)$...

%You need calculus in order to do quantitative analysis of how variables
%change over time (derivatives) or sum up all kinds of contributions that
%add up to a total (integration).

%The basic quantities that will appear in the equations of physics are 
%measured in meters [m], seconds [s], meters per second [m/s],  meters per second squared [m/s$^2$] and
%Newtons [N]. You should always keep in mind which units are being used (meters vs. kilometres) and
%\emph{what type} the type quantity you are manipulating (length vs. velocity). 
%%
%You should also be aware whether the quantities are regular numbers 
%or \emph{vectors} with multiple components.


%\subsection{Derivatives}
%
%a

\subsection{Integrals}
\label{3369022ed3e212789ba2790306add593}%% integrals
\begin{wrapfigure}{r}{0pt}
\includegraphics[width=110pt]{/Library/WebServer/Documents/miniref/data/media/calculus/integral_as_region_under_curve_t.png}
\end{wrapfigure}

An integral corresponds to the computation of an \dokuitalic{area}
under a curve $f(t)$ between two points:
\[
 A(a,b) \equiv \int_{t=a}^{t=b} f(t)\;dt.
\]
The symbol $\int$ is a mnemonic for \dokuitalic{sum}, 
since the area under the curve corresponds in some
sense to the sum of the values of the function $f(t)$
between $t=a$ and $t=b$.
The integral is the total amount of $f$ between $a$ and $b$.
%
%For certain functions, it is possible to find an anti-derivative function 
%$F(t)$ which is the ``running total'' of the total area under the curve. 
%The area under $f(t)$ between $a$ and $b$ is computed as the \dokuitalic{change} in $F(\tau)$:
%\[
% A(a,b) = F(b) - F(a).
%\]

%The anti-derivative function $F(\tau)$ is also known 
%as the \dokuitalic{indefinite integral} of $f(t)$ because it 
%corresponds to the integral calculation without the
%need to specify a \dokuitalic{definite} number for the upper limit:
%\[
% F(\tau) = \int^\tau_0 f(t)\;dt + F(0) = A(0,\tau) + F(0).
%\]
%We use the Greek letter $\tau$ (tau) to denote the input of $F$ since $t$
%is already used as the integration variable.

%The choice of $t=0$ as the starting point is 
%completely arbitrary.  We could have used a different
%starting point and obtained a different anti-derivative:
%\[
% F_2(\tau) = \int^\tau_{42} f(t)\;dt + F_2(42) = A(42,\tau) + F_2(42).
%\]
%Both $F$ and $F_2$ are anti-derivatives of the function $f$
%and contain all the information about the area under the curve
%\[
%  A(a,b) = F(b) - F(a) = F_2(b) - F_2(a).
%\]
%The effect of the choice of the starting point is to
%add or subtract a constant term to the anti-derivative.
%All anti-derivatives of the function $f(t)$ differ only by
%an additive constant factor $F=F_2 + C$.
%We call $C$ the integration constant, and its value
%is usually specified by external conditions.


Consider for example the constant function $f(t)=3$,
and let us find the expression $F(\tau)\equiv A(0,\tau)$
that corresponds to the area under $f(t)$ between 
$t=0$ and the time $t=\tau$.

\begin{wrapfigure}{r}{0pt}
\includegraphics[width=110pt]{/Library/WebServer/Documents/miniref/data/media/calculus/simple_integral_one_tikz-0.png}
\end{wrapfigure}

We can easily find this area because the region under the curve is rectangular:
\[ 
 \! F(\tau) \equiv A(0,\tau) =  \! \int_0^\tau \!\! f(t)\;dt  = 3 \tau.
\]
Indeed the area is equal to the height times the length of the 
base of the rectangle.


\begin{wrapfigure}{r}{0pt}
\includegraphics[width=110pt]{/Library/WebServer/Documents/miniref/data/media/calculus/simple_integral_two_tikz-0.png}
\end{wrapfigure}

Anther important calculation is the area under 
the function $g(t)=t$. We will compute $G(\tau)\equiv A(0,\tau)$,
which corresponds to the area under $g(t)$ between $0$ and $\tau$.
This area is also easily computed since the region under the curve 
is triangular:
\[
 G(\tau) \equiv A(0,\tau) = \int_0^\tau g(t) \; dt = \frac{\tau\times\tau}{2} = \frac{1}{2}\tau^2,
\]
since the area of a triangle is the product of the length of
the base times the height divided by two.


We were able to compute the above integrals thanks to the
simple geometry of the areas under the curves. Later on in
this book we will develop techniques for finding 
integrals of more complicated functions.
%In fact, there is an entire course, Calculus II,
%which is dedicated to the task of finding integrals.

What you need to remember for now that the integral 
of a function is the total amount of the function accumulated
during some time period.
You should try to remember the formulas:
\[
 \int_c^\tau a \;dt = a\tau + C, \qquad 
 \int_c^\tau at \;dt = \frac{1}{2}a\tau^2 + C,
\]
which correspond to the integral of a constant function
$f(t)=a$ and the integral of $f(t)=at$, a line with slope $a$.
%Note how we always add an additive constant term $+C$ when
%we state integral formulas. This is to remind  us that the 
%answer has an extra parameter (an additive constant term),
%which must be specified by the problem.
Note that each time you give a general integral formula
the answer will contain an additive constant term $+C$,
which depends on the starting point $t=c$, which 
we use for the area calculation. 
In the above examples we used $c=0$ as the initial
point so we had $C=0$.

Using the above formulas in combination,
you can now compute the integral under of 
an arbitrary line $f(t)=mt+b$ as follows:
\[
% F(\tau)=
%  \int_0^\tau f(t)\;dt =
  \int_c^\tau (mt + b)\;dt 
%  =  \int_0^\tau \!mt\;dt\  + \int_0^\tau \!b\;dt
 = \frac{1}{2}m\tau^2 + b \tau  + C,
\]
since the integral of the sum of two functions is the
sum of the integrals.

Now that you know about vectors and integrals, 
we can start our discussion on the laws of physics and
derive the equation of motion for a particle in free fall. 


%\subsection{Fundamental Theorem of Calculus}




\dokutitleleveltwo{Kinematics}
\label{508ac5264059de6a350383a9f1e87977}%% kinematics
Kinematics (from the Greek word \dokuitalic{kinema} for \dokuitalic{motion}) is the study of trajectories of moving objects.  
The equations of kinematics can be used to calculate how long a ball thrown upwards will stay in the air, 
or to calculate the acceleration needed to go from 0 to 100 km/h in 5 seconds.

\dokutitleleveltree{Concepts}
\label{408d82ccb63ca28fa1665f5ee146b453}%% position_velocity_and_acceleration_revisited

The key notions used to describe the motion of an objects are:
\begin{itemize}
\dokuitem  $t$: the time, measured in seconds [s].
\dokuitem  $x(t)$: the position of an object as a function of time -- also known as the equation of motion. Measured in meters [m].
\dokuitem  $v(t)$: the velocity of the object as a function of time. [m/s]
\dokuitem  $a(t)$: the acceleration of the object as a function of time. [m/s$^2$]
\dokuitem  $x_i=x(0), v_i=v(0)$: the starting position and velocity.
\end{itemize}


The position, velocity and acceleration functions ($x(t)$, $v(t)$ and $a(t)$) are connected. 
They all describe different aspects of the same motion.
The function $x(t)$ is the main function since it describes the position of the object at all times.
The velocity function describes the change in the position over time, hence it is measured in [m/s].
The acceleration function describes how the velocity changes over time.
A constant positive acceleration means the velocity of the motion 
is steadily increasing, like when you press the gas pedal in your car.
A constant negative acceleration means the velocity is steadily decreasing,
like when pressing the brake pedal.

{\bf If you know the exact function $x(t)$}, then you can compute its \emph{derivative}
and obtain the velocity function $v(t)$. You can obtain the acceleration function $a(t)$
by computing the derivative of the velocity $v(t)$:
\[
  a(t) \overset{\frac{d}{dt} }{\longleftarrow} v(t) \overset{\frac{d}{dt} }{\longleftarrow} x(t).
\]

Alternately, {\bf if you know the acceleration function $a(t)$}, you can use
integration in order to obtain the velocity function $v(t)$ and then integrate
the velocity function in order to obtain the position function $x(t)$:
\[
 a(t) \overset{\ \int\!dt}{\longrightarrow} v(t) \overset{\ \int\!dt }{\longrightarrow} x(t).
\]
which means that we start from the acceleration $a(t)$ and use integration with
respect to time to obtain the velocity $v(t)$. If we integrate the velocity 
we obtain the position function $x(t)$.

Recall that the integral is the calculation of the total 

This is 
The main subject in this section is how to use 



\subsection{Uniform acceleration motion}

 (UAM)
 
This is how the \dokuitalic{equations of motion} are derived. 
Assuming that the acceleration is constant in time $a(t) =a$,
we can calculate the velocity by \dokuitalic{adding up} all the acceleration (integrating)
to obtain the change in the velocity
\[
 v(t) = \int a(t) \ dt = \int a \ dt = at + v_i.
\]

\[
 x(t) = \int v(t) \; dt = \int (at+v_i) \; dt = \frac{1}{2}at^2 + v_it + x_i.
\]




\dokutitlelevelfour{Formula}

If the object undergoes a \dokuitalic{constant} acceleration $a(t)=a$,
like your car if you floor the \dokuitalic{accelerator} pedal, then
the equations of motion are:

\begin{align}
  x(t) &= \frac{1}{2}at^2 +  v_i t + x_i, \\
  v(t) &= at + v_i, \\
  a(t) &= a.
\end{align}

There is also another very useful equation to remember:
\[
 v(t)^2 = v_i^2 + 2a[x(t)- x_i],
\]
which is usually written
\be
 v_f^2 = v_i^2 + 2a\Delta x.
\ee

That is it. Memorize these equations, plug-in the right numbers,
and you can solve any kinematics problem humanly imaginable.
Chapter done.



%

%
%%To carry out these calculations we need to know the \dokuitalic{equation of motion} and the \dokuitalic{initial conditions} (like how fast you threw the ball up \(v_{i}\)) and then carry out some simple algebra steps to calculate anything we want. No seriously, this entire chapter boils down to three equations. 
%%It is all about the plug-number-into-equation skills.

%
%\dokutitleleveltree{Concepts}
%\label{ff4e01de0bd379280a0157bd102cc5f0}%% concepts

%The basic concepts of kinematics are:

%

%\begin{itemize}
%\dokuitem  \(t\): time, measured in seconds [s].
%\dokuitem  \(x\): the \dokuitalic{position} of an object with respect to a coordinate system, measured in meters [m].
%\dokuitem  \(x(t)\): the position as a function of time, also known as the equation of motion.
%\dokuitem  \(v(t)\): speed measured in meters per second [m/s].
%\dokuitem  \(a(t)\): acceleration measured in meters per second squared [m/s\^{ }2].
%\dokuitem  \(x_0=x(0)\): the position at \(t=0\).
%\dokuitem  \(v_{0}=v(0)\): the velocity at \(t=0\).
%\end{itemize}

%When solving some problem, where we calculate the motion of an object that starts
%form an \dokuitalic{initial} point an goes to a \dokuitalic{final} point we will use the following terminology:

%\begin{itemize}
%\dokuitem  \(t_i\): initial time (the beginning of the motion).
%\dokuitem  \(t_f\): final time (when the motion stops).
%\dokuitem  \(x_i\): initial position.
%\dokuitem  \(x_f\): final position.
%%\dokuitem  \(\vec{d} = \Delta x = x_f-x_i\): delta x  = change in position = displacement.
%%\dokuitem  \(d = |\Delta x| = |x_f-x_i|\): distance traveled = absolute value of displacement.
%\dokuitem  \(v_i\): initial velocity of the object.
%\dokuitem  \(v_f\): final velocity of the object.
%\end{itemize}

%\dokutitleleveltree{Formulas}
%\label{51d24e1edefe34e683025dbba5c6eed6}%% formulas
%There are basically three equations that you need to be aware
%of for this entire chapter.
%Together, these three equations fully describe all the aspects
%of motion with constant acceleration. 

%

%
%\dokutitlelevelfour{Uniform acceleration motion (UAM)}

%If the object undergoes a \dokuitalic{constant} acceleration \(a_{const}=a\),
%like your car if you floor the \dokuitalic{accelerator} pedal, then
%the equations of motion are:

%\[
%  a(t) = a,
%\]
%\[
%  v(t) = at + v_0,
%\]
%\[
%  x(t) = \frac{1}{2}at^2 + v_0t + x_0.
%\]

%There is also another very useful equation to remember:
%\[
% v(t)^2 = v_0^2 + 2a[x(t)- x_0],
%\]
%which is usually written
%\[
% v_f^2 = v_i^2 + 2a\Delta x.
%\]

%That is it. Memorize these equations, plug-in the right numbers,
%and you can solve any kinematics problem humanly imaginable.
%Chapter done.

%
%\dokutitlelevelfour{Uniform velocity motion (UVM)}

%The special case where there is zero acceleration (\(a=0\)),
%is called \dokuitalic{uniform velocity motion} or UVM.
%%The velocity is uniform (constant), because there is no acceleration.
%%The following three equations describe the motion of the object
%%under uniform velocity:
%\[
%  a(t) = 0,
%\]
%\[
%  v(t) = v_0,
%\]
%\[
%  x(t) = v_0t + x_0.
%\]

%As you can see, these are really the same equations as above, but
%because \(a=0\), some terms are missing.

%
%\dokutitleleveltree{Explanations}
%\label{678d5f6d14642c24a1e4bceffedbe407}%% explanations

%\dokutitlelevelfour{Calculus}
%The functions \(x(t)\), \(v(t)\) and \(a(t)\) are connected. They all describe
%different aspects of the same motion. We have:
%\[
%  x(t) \overset{\frac{d}{dt} }{\longrightarrow} v(t) \overset{\frac{d}{dt} }{\longrightarrow} a(t),
%\]
%which means that starting from the position function, \(x(t)\), we can use the derivative
%operation to obtain the velocity and the acceleration.
%We can also use this relationship in the other direction:
%\[
%  x(t) \overset{\ \int\!dt }{\longleftarrow} v(t) \overset{\ \int\!dt }{\longleftarrow} a(t),
%\]
%which means that we start from the acceleration \(a(t)\) and use integration with
%respect to time to obtain the velocity \(v(t)\). If we integrate the velocity 
%we obtain the position function \(x(t)\).

%%This is how the \dokuitalic{equations of motion} are derived. 
%%Assuming that the acceleration is constant in time \(a(t) =a\),
%%we can calculate the velocity by \dokuitalic{adding up} all the acceleration (integrating)
%%to obtain the change in the velocity
%%\[
%% v(t) = \int a(t) \ dt = \int a \ dt = at + v_i.
%%\]
%%If you are moving at velocity \(100\)[m/s] and you accelerate
%%at a rate of \(10\)[m/s\^{ }2] during 3 seconds, then your final 
%%velocity at the end of the three seconds will be
%%\[
%% v(3)  = \int_0^3 10\ dt + 100 = 10(3) + 100 = 130 [m/s].
%%\]



\paragraph{Moroccan example}

Suppose your friend wants to send you a ball wrapped in aluminum 
foil from his balcony, which is located on the 14th floor (height of $44.145$[m]). 
At  $t=0$[s] he \dokuitalic{throws} the ball straight down with an initial velocity 
of $10$[m/s]. How long does it take for the ball to hit the ground?

Assume that the $y$-axis measuring distance upwards starting from the ground floor.
We know that the balcony is located at a height of $y_i=44.145$[m],
and that at $t=0$[s] the ball starts with $v_i=-10$[m/s].
The initial velocity is negative, because it points in the opposite direction to the $y$-axis.
We know that there is an acceleration due to gravity of $a_y=-g=-9.81$ [m/s$^{2}$].

We start by writing out the general UAM equation:
\[
 y(t) = \frac{1}{2}a_yt^2 + v_i t + y_i.
\]
We want to find the time when the ball will hit the ground,
so $y(t)=0$. To find $t$,
we plug in all the known values into the general equation:
\[
 y(t) = 0 = \frac{1}{2}(-9.81)t^2  -10 t + 44.145,
\]
which is a quadratic equation in $t$.
First rewrite the quadratic equation into the standard form:
\[
 0 =  \underbrace{4.905}_a t^2 + \underbrace{5.0}_b \ t - \underbrace{44.145}_c,
\]
and then solve using the quadratic equation:
\[
 t_{f} = \frac{-b \pm \sqrt{ b^2 - 4ac }}{2a} = \frac{-5 \pm \sqrt{ 25 +  866.12}}{9.81} =  2.53 \text{ [s]}.
\]
We ignored the negative-time solution.
%Comparing with the first Moroccan example, we see that the answer makes sense: 
%throwing a ball downwards will make it fall to the ground faster than just dropping it.














\section{Projectile motion}

The basic concepts of kinematics  in two dimensions are:

\begin{itemize}
\dokuitem  \(t\): time, measured in seconds.
\dokuitem  \(\vec{r}(t)\equiv (x(t),y(t))\): the position vector 
\dokuitem  \(\vec{v}(t) \equiv (v_x(t), v_y(t) ) \): the velocity as a function of time.
\dokuitem  \(\vec{a}(t) \equiv (a_x(t), a_y(t) ) \): the acceleration as a function of time.
\end{itemize}

When solving some problem, where we calculate the motion of an object that starts
form an \dokuitalic{initial} point an goes to a \dokuitalic{final} point we will use the following terminology:


\begin{itemize}
\dokuitem  \(t_i=0\): initial time (the beginning of the motion).
\dokuitem  \(t_f\): final time (when the motion stops).
\dokuitem  \(\vec{v}_{i}=\vec{v}(t_i)=(v_x(0),v_y(0))=(v_{xi},v_{yi})\): the initial velocity at \(t=t_i\).
\dokuitem  \(\vec{r}_i=\vec{r}(0)=(x(0),y(0))=(x_i,y_i)\): the initial position at \(t=0\).
\dokuitem  \(\vec{r}_f=\vec{r}(t_f)=(x(t_f),y(t_f))=(x_f,y_f)\): the final position at \(t=t_f\).
\end{itemize}


\begin{figure}
\hspace{-0.3cm}
\includegraphics[width=0.5\textwidth]
{/CurrentPorjects/Minireference/miniref_figures/plots_and_diagrams/projecticle-concepts_tikz.pdf}
\end{figure}



\dokutitleleveltree{Formulas}
\label{51d24e1edefe34e683025dbba5c6eed6}%% formulas

\dokutitlelevelfour{Motion in two dimensions}

Sometimes you have to describe both the \(x\) and the \(y\) coordinate
of the motion of a particle:
\[
 \vec{r}(t)=(x(t), y(t)).
\]
We choose \(x\) to be the horizontal component of the projectile motion,
and \(y\) to be its height.

The velocity of the projectile will be:
\[ 
 \vec{v}(t) = \frac{d}{dt}\left(\vec{r}(t)\right) = \left(\frac{dx(t)}{dt}, \frac{dy(t)}{dt} \right) = (v_x(t),v_y(t)),
\]
and the initial velocity is:
\[ 
  \vec{v}_i = \vec{v}(t_i) = |\vec{v}_i|\angle \theta = (v_x(t_i), v_y(t_i)) = (v_{ix}, v_{iy})=
(|\vec{v}_i|\cos\theta, |\vec{v}_i|\sin\theta).
\]

The acceleration of the projectile will be
\[ 
 \vec{a}(t) = \frac{d}{dt}\left(\vec{v}(t)\right) = (a_x(t),a_y(t)) = (0,-9.81).
\]
Note how we have zero acceleration in the \(x\) direction so we can use
the UVM equations of motion for \(x(t)\) and \(v_x(t)\).
In the \(y\) direction we have a uniform downward acceleration due to gravity.


\dokutitlelevelfour{Projectile motion}

The equations of motion of a projectile are the following.
First in the \(x\) direction we have uniform velocity motion (UVM):
\begin{align*}
 x(t)   & = v_{ix}t + x_i,  \\
 v_x(t) & =v_{ix}.
\end{align*}

In the $y$ direction, you have the constant pull of gravity downwards 
which gives us a uniformly accelerated motion (UAM):
\begin{align*}
 y(t) & = \frac{1}{2}(-9.81)t^2 + v_{iy}t + y_i, \\
 v_y(t) & = -9.81 t + v_{iy}, \\
 v_{yf}^2 & = v_{yi}^2 + 2(-9.81)(\Delta y).
 \end{align*}




\dokutitleleveltwo{Forces}
\label{2c05161a7e67e946ddc889571aac4e89}%% forces

Like a shepherd who brings in a stray sheep back, we need to rescue the word \dokuitalic{force} and give it precise meaning. In physics force means something very specific. Not ``the force'' from Star Wars, not the ``force of public opinion'', and not the \dokuitalic{force} in the battle of good versus evil.

Force in physics has a precise meaning as an amount of push or pull exerted on an object. 
Force is a vector. We measure force in Newtons [N],
and we can use it in equations and solve for it just like any other unknown.
In this section we will explore all the different kinds of forces. 


\dokutitleleveltree{Concepts}
\label{ff4e01de0bd379280a0157bd102cc5f0}%% concepts

\begin{itemize}
\dokuitem  \(\vec{F}\): a force. This is something the object ``feels'' as a pull or a push. Force is a vector, so you must always keep in mind the direction in which the force \(\vec{F}\) acts.
\dokuitem  \(k,G,m,\mu_s,\mu_k,\ldots\): parameters on which the force \(F\) may depend. Ex: the heavier an object is (has large \(m\) parameter), the larger its gravitational pull will be: \(\vec{W}=-9.81m\hat{z}\), where \(\hat{z}\) points towards the sky.
\end{itemize}

\dokutitleleveltree{Kinds of forces}
\label{517db664a569f42f769556092e40d53e}%% kinds_of_forces

We next list all the forces which you are supposed to know about for a standard physics class
and define the relevant parameters for each kind of force.
You need to practice exercises using each of these forces, until you start to \dokuitalic{feel} how they act.


\dokutitlelevelfour{Gravitation}

Manifestations of the gravitational pull of the planet Earth on massive objects:


\begin{itemize}
\dokuitem  \(M\): mass of the earth. \(M=5.9721986 \times 10^{24}\) [kg].
\dokuitem  \(m\): mass of an object.
\dokuitem  \(\vec{W}=\vec{F}_g\): The weight (the force on a object due to gravity).
\dokuitem  \(G\): Gravitational constant = \(6.67 \times 10^{-11}\) [\(\frac{Nm^2}{kg^2}\)].
\dokuitem  \(\vec{F}_g=\frac{GMm}{r^2}\): Force of gravity between two objects of mass \(M\) and \(m\) respectively. Measured in Newtons [N].
\dokuitem  \(\vec{F}_g=gm\) (downward): The force of gravity on the surface of the earth, where \(g=\frac{GM}{r^2} \approx9.81\ldots\) [N/kg]=[m/s\^{ }2].
\end{itemize}

The famous one-over-arr-squared law that describes the gravitational
pull between two objects is:
\[
  F_g=\frac{GMm}{r^2}.
\]
You will rarely use it, but it is extremely important as this is where all of mechanics began.
This was Newton's big discovery. All the rest of mechanics is simple calculus,
but this equation is \dokuitalic{real} physics.
It tells us something about how the Universe works.


At the surface of the earth:
\[
 \vec{F}_g = \frac{GMm}{r^2} = \underbrace{\left(\frac{GM}{r^2}\right)}_gm = \vec{g}m = \vec{W},
\]
where the weight \(\vec{W}\) of an object is a vector
that points towards the centre of the earth, and \(g=9.81\)[m/s\^{ }2].


\dokutitlelevelfour{Force of a spring}

\begin{itemize}
\dokuitem  \(\vec{F}_s=-kx\): The force (pull or push) of a spring that is displaced  (stretched or compressed) by \(x\) meters. The constant \(k\) [N/m] is a measure of the  \dokuitalic{strength} of the spring, or its stiffness.
\end{itemize}

\dokutitlelevelfour{Tension in a rope}

\begin{itemize}
\dokuitem  \(\vec{T}\): Tension in a rope. Tension is always pulling \dokuitalic{away} from an object:  you can't push a dog on a leash.
\end{itemize}

\dokutitlelevelfour{Contact force}

\begin{itemize}
\dokuitem  \(\vec{C}\): Contact force between two rigid objects. We generally brake-up contact forces into two components: perpendicular and parallel to the contact surface.
\dokuitem  \(\vec{N}\equiv\vec{C}_{\perp}\): Normal force: the force between two surfaces. Normal is a mathematically precise way to say ``perpendicular to a surface''. Intuitively, you can think of \(\vec{N}\) as the force that a surface exerts on an object to keep it where it is. The reason why my coffee mug does not fall to the floor, is that the table exerts a normal force on it keeping in place.
\dokuitem  \(\vec{F}_f\equiv\vec{C}_{||}\): Force of friction between two surfaces. There are two kinds, both of which are proportional to the normal force between the surfaces: \\  Kinetic: \[
    F_{fk}=\mu_k|\vec{N}|.
    \] Static: \[
     F_{fs}=\mu_s|\vec{N}|.
    \]
\end{itemize}

\dokutitlelevelfour{Two kinds of friction forces}

\begin{itemize}
\dokuitem  \(\vec{F}_{fs}=\mu_s|\vec{N}|\): Static force of friction, for objects that are not moving. 
\dokuitem  \(\mu_s\): The static coefficient of friction.  ex: 0.3.  It describes the \dokubold{maximum} amount of friction that  can exist between two objects. If a horizontal force exists greater than \(F_{fs} = \mu_s N\), then the object will start to slip.     
\dokuitem  \(\vec{F}_{fk}=\mu_k|\vec{N}|\): Kinetic force of friction acts when two objects are sliding relative to each other. It always acts in a direction opposing the motion. 
\dokuitem  \(\mu_k\): Kinetic coefficient of friction. ex: \(\mu_k=0.1\). Dimensionless.   it is just the ratio that describes how much friction an object feels for a given amount of normal force.
\end{itemize}

%\dokutitleleveltree{Discussion}
%\label{bd8bc36eb41bc90c585ae7e902e9e284}%% discussion
%Ok so what is mechanics all about? You should know by now, since you have already 
%learned about forces and about acceleration, velocity and position.
%Doesn't mechanics have something to do with \(F=ma\)?

%You can think of the different forces as the ``causes of motion'',
%and the effect is the acceleration \(a\).
%Once you have the acceleration, you can use the basic kinematics equations like 
%\(x(t)=\frac{1}{2}at^2+v_{i}t + x_i\) which describe the motion
%at all times $t$.

%The causes and effects of the motion of bodies are represented in this diagram:

%\begin{figure}[h]
%\centering
%\includegraphics[width=320pt]{/Library/WebServer/Documents/miniref/lib/plugins/tmp/texitimages/texitei6f7D.png}
%\end{figure}

%
%To understand any physics problem you must first check which kinds of forces are relevant,
%make a force diagram and add up all the forces to obtain \(\vec{F}_{net}\), the net force on the object.
%Then you use \(\vec{a}=\frac{\vec{F}_{net}}{m}\) and you have your acceleration. 
%Problem solved,  because once you have \(a\) you just need to plug
%it into equations of motion:
%\[
% \vec{a}(t) = \vec{a},
%\]
%\[
% \vec{v}(t) = \vec{a}t+\vec{v}_i,
%\]
%\[
% \vec{r}(t) = \frac{1}{2}\vec{a}t^2+\vec{v}_it + \vec{r}_i.
%\]
%There you have it. All of undergraduate level mechanics boils down to your understanding and \dokuitalic{usage} of the equation \(F=ma\).

%\dokubold{Disclaimer:} Of course physics is more complicated than that, but what you are learning is an important
%and very useful subset. Namely, the above three equations make the assumption that the acceleration is constant,
%which in turn assumes that the balance of the forces acting on the object will always remain the same.
%We assume that during the entire time between \(t_i\) and \(t_f\) the force \(\vec{F}_{net}\) stays constant.
%In a \dokuitalic{lot} of situations this assumption is true.

%In more complicated situations you could have \(\vec{F}_{net}\) changing over time and thus
%\(a(t)\) changing over time too. You would have to start from \(a(t)\) and do the integration
%thing twice on \(a(t)\) to get the \(x(t)\) for the motion. 

%Sometimes the net force on the object \(\vec{F}_{net}\) might be constant, 
%but the mass of the object may change
%over time (like a multi-stage rocket losing its booster tanks) 
%so you would still have \(a(t)=\frac{F_{net}}{m(t)}\)
%changing with time because \(m(t)\) would be changing.
%Again, for those situations you have to go back to calculus and integrate \(a(t)\)
%twice to get the \(x(t)\) function.

%


\dokutitleleveltwo{Force diagrams}
\label{dac25f0c5f2078a480e480009deb52c9}%% force_diagrams

%Welcome to force-accounting 101. In this section we will learn how 
%identify all the forces acting on an object and predict the resulting
%acceleration.

Newton's 2nd law says that the \dokuitalic{net} force on an object
causes an acceleration:
\[
 \sum \vec{F}=\vec{F}_{net} = m\vec{a},
\]
so finding the net force must be a pretty important thing.


\dokutitleleveltree{Concepts}
\label{ff4e01de0bd379280a0157bd102cc5f0}%% concepts
Newton's second law is a relationship between these three concepts:


\begin{itemize}
\dokuitem  \(m\): the \dokuitalic{mass} of an object. 
\dokuitem  \(\vec{F}\): vector used to denote any kind of \dokuitalic{force}. 
\dokuitem  \(\vec{a}\): the \dokuitalic{acceleration} of the object. 
\end{itemize}


What types of forces are there in force diagrams?



\begin{itemize}
\dokuitem  \(\vec{W}\equiv\vec{F}_{gravity}=m\vec{g}\): The \dokuitalic{weight}. This is the force on a object due to its gravity. The gravitational pull \(\vec{g}\) always points downwards  towards the center of the earth. \(g=9.81\) [N/kg].
\dokuitem  \(\vec{T}\): Tension in a rope. Tension is always pulling \dokuitalic{away} from the object.
\dokuitem  \(\vec{N}\): Normal force  the force between two surfaces.
\dokuitem  \(\vec{F}_{fs}=\mu_s|\vec{N}|\): Static force of friction. 
\dokuitem  \(\vec{F}_{fk}=\mu_k|\vec{N}|\): Kinetic force of friction. 
\dokuitem  \(\vec{F}_{spring}=-kx\): The force (pull or push) of a spring that is displaced (stretched or compressed) by \(x\) meters. 
\end{itemize}

\dokutitleleveltree{Formulas}
\label{51d24e1edefe34e683025dbba5c6eed6}%% formulas

\dokutitlelevelfour{Newton's 2nd law}

The sum of the forces acting on an object,
divided by the mass gives you the acceleration of the object:
\[
 \sum F = \vec{F}_{net}= m\vec{a}.
\]


%
%In what follows, you will be asked a countless number of times to
%\[
% \text{Find the component of } \vec{F} \text{ in the ? direction. }
%\]
%Which is another way of asking you to find the number \(v_?\).

%The answer is usually equal to the length \(|\vec{F}|\) multiplied by
%either \(\cos\) or \(\sin\) and sometimes \(-1\) all 
%\dokubold{depending on way the coordinate system is chosen}.
%So don't guess. Look at the coordinate system. If the vector points
%in the direction where \(x\)-increases, then \(v_x\) should be a positive
%number. If it points in the opposite direction, then \(v_x\) should be 
%negative.

%Vector components are important, because to add forces \(\vec{F}_1\)
%and \(\vec{F}_2\) you have to add their components:
%\[
% (F_{1x},F_{1y}) + (F_{1x},F_{2y}) 
% = \vec{F}_1 + \vec{F}_2 
% = (F_{1x}+F_{2x},F_{1y}+F_{2y})
% = \vec{F}_{net}.
%\]
%Instead of dealing with vectors in the bracket notation as above,
%when solving force diagrams it is easier to simply write the \(x\)-equation
%on one line, and the \(y\)-equation on a separate line.
%\[
% F_{netx} = F_{1x}+F_{2x},
%\]
%\[
% F_{nety} = F_{1y}+F_{2y}.
%\]

It is a good idea to always write those two equations together
as a block  so it remains clear that  you are talking
about the same problem, but the first row represents the \(x\)-dimension 
and the second row represents the \(y\)-dimension.


%\dokutitlelevelfour{Force check}

%It is important to account for \dokuitalic{all} the forces acting on an object.

%Any object with mass on the surface of the earth will feel a 
%gravitational pull of magnitude \(\vec{F}_{gravity}=\vec{W}=m\vec{g}\)
%downwards. 

%Then you have to think about which of the other forces
%you know might be present: \(\vec{T}\), \(\vec{N}\), \(\vec{F}_{f}\), \(\vec{F}_{spring}\).

%Anytime you see a rope tugging on the object  you know you there must be some tension \(\vec{T}\), which is a force vector pulling on the block. 

%Anytime you have an object sitting on a surface, the surface will push back with a \dokuitalic{normal} force \(\vec{N}\). 

%If the object is sliding on the surface there will be a force of friction acting acting against the direction of the motion:
%\[
% F_{fk}=\mu_k|\vec{N}|.
%\]

%If the object is not moving, then you have to use \(\mu_s\) in
%the friction force equation, to get the maximum static 
%friction force that the contact between the object and the ground
%can support before the object starts to slip:
%\[
% \max\{ F_{fs} \}=\mu_s|\vec{N}|.
%\]

%If you see a spring that is either stretched or compressed by the object, then you must account for the spring force. The force of a spring is \dokuitalic{restorative}: it always acts against the deformation you are making to the spring. If you stretch it by \(x\)[cm], then
%it will try to pull itself back to its normal length with a force of:
%\[
% \vec{F}_s = -kx \hat{\imath}.
%\]
%The constant of proportionality \(k\) is called the \dokuitalic{spring constant} and in the above
%example would be measured in [N/cm]. 


\dokutitleleveltree{Recipe for solving force diagrams}
\label{31bb2b3f8fca76563afa16cf8fbccb90}%% recipe_for_solving_force_diagrams

\begin{enumerate}\dokuitem  Draw a diagram centred on the object and draw all the forces acting on it.
\dokuitem  Choose a coordinate system, and indicate clearly what you will call the \(x\)-direction, and what you will call the \(y\)-direction. All \dokubold{equations are expressed \dokuitalic{with respect to} this coordinate system}.
\dokuitem  Write down this ``template'': \[ \sum F_x = \qquad \qquad \qquad = ma_x \]   \[ \sum F_y = \qquad \qquad \qquad = ma_y \]
\dokuitem  Fill the first line by finding the \(x\)-components of each force acting on the object.
\dokuitem  Fill the second line by finding the \(y\)-components of each force acting on the object.
\dokuitem  Consistency checks:
\begin{enumerate}\dokuitem  Check signs. If the force in the diagram is acting in the \(x\)-direction  then its component must be positive. If the force is acting in the  opposite direction to \(\hat{x}\), then its component should be negative.
\dokuitem  Verify that whenever \(F_x \propto \cos\theta\), then \(F_y \propto \sin\theta\).  If instead we use an angle \(\phi\) defined with respect to the \(y\)-axis we would have \(F_x \propto \sin\phi\), and \(F_y \propto \cos\phi\). 
\end{enumerate}

\dokuitem  Solve the two equations finding the one or two unknowns.  If there are two unknowns, you may need to solve two equations simultaneously by isolation and substitution.
\end{enumerate}

Force diagrams are best explained through examples.





\dokutitleleveltwo{Momentum}
\label{3a749f8e94241d303c81e056e18621d4}%% momentum

%Say you have a 1g piece of paper and a 1000kg car moving at the same speed 100km/h.
%Which would you rather get hit by? In physics the ``amount moving stuff'' has a precise
%name: \dokuitalic{momentum}, and it is denoted with \(\vec{p}\).
%You see, your gut feeling about the paper and the car is correct: in collisions it is the momentum that counts. In this section, we will add some mathematical precision to that gut feeling. 

The momentum is equal to the velocity of the moving object multiplied by the object's mass
 (\(\vec{p} = m\vec{v}\)).
Therefore, since the car weights \(1000\times1000=10^{6}\) times more than the piece of paper, it has \(10^6\) times more momentum when moving at the same speed. 
A collision with it will ``hurt'' that much more.

%Note that momentum is a vector, so we will have to do a lot of that length-and-direction-to-components transformation stuff:
%\[
% (p_x, p_y) = \vec{p} = (|\vec{p}|\cos\theta, |\vec{p}|\sin\theta) = |\vec{p}| \angle \theta,
%\]
%and also converting backwards from component notation to magnitude-direction:
%\[
% |\vec{p}| = \sqrt{ p_x^2 + p_y^2 }, \qquad \theta = \tan^{-1}\!\left( \frac{ p_{y} }{ p_{x} } \right).
%\]


\dokutitleleveltree{Concepts}
\label{ff4e01de0bd379280a0157bd102cc5f0}%% concepts

\begin{itemize}
\dokuitem  \(m\): the \dokuitalic{mass} of the moving object.
\dokuitem  \(\vec{v}\): the \dokuitalic{velocity} of the moving object.
\dokuitem  \(\vec{p}=m\vec{v}\): the \dokuitalic{momentum} of the moving object.
\end{itemize}

\dokutitleleveltree{Formulas}
\label{51d24e1edefe34e683025dbba5c6eed6}%% formulas

\dokutitlelevelfour{Definition}

The momentum of an object is the mass of the object times its velocity:
\[
 \vec{p} = m\vec{v}.
\]

If you speed is \(\vec{v}=(20,0,0)\)[m/s], which is 
equivalent to saying ``20[m/s] in the \(x\)-axis direction'', and your mass is 100kg then your momentum is \(\vec{p}=(2000,0,0)\)[kg*m/s].


%\dokutitlelevelfour{Newton's first law}

%Newton's first law states that whenever there is no acceleration (\(\vec{a}=0\)), 
%an object will maintain a constant velocity. 
%This is kind of obvious if you know Calculus, since \(\vec{a}\) is the derivative of \(\vec{v}\).
%For example, if an object is stationary and there are no forces on it to cause it to accelerate, then it will stay stationary.
%If an object was moving with velocity \(\vec{v}\) and there is no acceleration, then it will keep moving with velocity \(\vec{v}\) forever.
In the absence of acceleration, objects will conserve their velocity:
\[
  \vec{v}_{in}= \vec{v}_{out}.
\]
This is equivalent to saying that objects conserve their momentum (just multiply the velocity by the mass  if the mass stays constant and the velocity stays constant, then the momentum must stay constant).


\dokutitlelevelfour{Conservation of momentum}

More generally, if you have a situation with multiple moving objects,
you can say that the ``overall momentum'', i.e., the sum of the momenta
of all the particles stays constant:
\[
 \sum \vec{p}_{in} =  \sum \vec{p}_{out}.
\]

This is amazingly powerful stuff, and one of the furthest reaching 
laws of physics. Whatever momentum comes into a collision must come out.
%This is a kind of Le Chatelier's principle 
%``rien ne se cr�e, et rien ne disparait tout se transforme''.
%Motion cannot simply appear or disappear, it can only be exchanged between systems.





\dokutitleleveltwo{Energy}
\label{05e7d19a6d002118deef70d21ff4226e}%% energy

Instead of thinking about velocities \(v(t)\) and motion trajectories \(x(t)\), we can solve physics problems using \dokuitalic{energy} calculations.
%The key idea in this section is the principle \dokuitalic{total energy conservation}, which tells us
%that the sum of the initial energies is equal to the sum of the final energies.
%We will define precisely the different kinds of energy, and then learn the rules of converting one energy into another.


\dokutitleleveltree{Concepts}
\label{ff4e01de0bd379280a0157bd102cc5f0}%% concepts

The concepts of energy come up in several different contexts.

\noindent
Moving objects:
\begin{itemize}
\dokuitem  \(m\): the mass of an object.
\dokuitem  \(v\): the velocity.
\dokuitem  \(E_K=K\): kinetic energy \( = \frac{1}{2}mv^2\).
\end{itemize}

\noindent
Moving objects by force:


\begin{itemize}
\dokuitem  \(\vec{F}\): the force needed to move the object.
\dokuitem  \(\vec{d}\): the displacement of the object. How far it moved. 
\dokuitem  \(W\): work done to move the object \( = \vec{F}\cdot \vec{d}\).
\end{itemize}

Gravity:


\begin{itemize}
\dokuitem  \(g\): gravitational acceleration on the surface of earth. 9.81 [m/s\^{ }2].
\dokuitem  \(h\): height of an object.
\dokuitem  \(U_g\): Gravitational potential energy  \(= mgh\).
\end{itemize}

Springs:


\begin{itemize}
\dokuitem  \(k\): spring constant. Measured in [N/m].
\dokuitem  \(x\): spring displacement from the relaxed position. If the spring is stretched then \(x>0\), and if it is compressed then \(x<0\).
\dokuitem  \(U_{s}\): Spring potential energy \(=  \frac{1}{2}kx^2\).
\end{itemize}

There are all kinds of other forms of energy: electric energy,
sound energy, thermal energy, etc. 
In this section we will focus on the types of \dokuitalic{mechanical} energy.


\dokutitleleveltree{Formulas}
\label{51d24e1edefe34e683025dbba5c6eed6}%% formulas
Kinetic energy:
\[
 K=\frac{1}{2}mv^2
\]


Work:
\[
  W = \int \vec{F}(x) \cdot dx 
\]
for constant force:
\[
  W = \vec{F}\cdot \vec{d} = |F||d|\cos\theta.
\]


Gravitational potential energy:
\[
  U_g = mgh,
\]
which is the energy you have because of your height.

Spring energy:
\[
 U_s = \frac{1}{2}kx^2.
\]


Conservation of energy:
\[
  \sum E_{in} + W_{in}  = \sum E_{out}  + W_{out}.
\]

%
%\dokutitleleveltree{Explanations}
%\label{678d5f6d14642c24a1e4bceffedbe407}%% explanations

%Energy is measured in Joules \([J]\).
%The dimension of energy [J] is related
%to other dimension you are familiar with:
%\[
% \text{ [J] = [N][m] = } \text{[kg]}[\text{m}^2][\text{s}^{-2}].
%\]
%You can check the second equality from \(F=ma\), which
%has dimensions [N] = [kg m/s\^{ }2].

%
%\dokutitlelevelfour{Kinetic energy}

%A moving object has energy,
%which we call \dokuitalic{kinetic} energy from the Greek word for motion \dokuitalic{kinema}.
%The kinetic energy is given by
%\[
% K=\frac{1}{2}mv^2.
%\]
%Note that velocity \(v\) and speed \(|v|\) are not the same as energy. Suppose you have two objects of the same mass and one is moving twice faster than the other. The faster object will have twice the velocity, but four times more kinetic energy.

%
%\dokutitlelevelfour{Work}

%When hiring someone to help you move, you have to pay them for the \dokuitalic{work} they do.
%Work is the product of how much force is necessary to move your furniture and
%the distance of the move.
%The more force, the more work there will be for a fixed displacement.
%The more displacement (think moving to south shore versus moving next door) the 
%more money the movers will ask for.

%The amount of work a force \(\vec{F}(x)\) (possibly changing) will produce when
%moved along some path \(p\) is given by:
%\[
% W = \int_p \vec{F}(x) \cdot d\vec{x}.
%\]

%If the force is constant, this expression simplifies to:
%\[
%  W = \int_0^d \vec{F}(x)\cdot d\vec{x} = \vec{F}\cdot\int_0^d d\vec{x}  = \vec{F}\cdot \vec{d} = |F||d|\cos\theta,
%\]
%where the factor of \(\cos\theta\) comes from the dot product.
%Indeed, we only want to account for the part of \(\vec{F}\) that is pushing in the direction of the displacement \(\vec{d}\).

%
%\dokutitlelevelfour{Potential energy is stored work}

%Some kinds of work are just a waste of your time, like working
%in a bank for example. You work and you get your paycheque, but nothing
%remains with you.
%Other kinds of work leave with you some \dokuitalic{resource} at the end of the work day.
%Maybe you learn something, or you network with a lot of good people.

%In physics, we make a similar distinction. Some types of work like
%work against friction are called \dokuitalic{dissipative} since they just waste energy.
%Other kinds of work are called \dokuitalic{conservative} since the work you do
%is not lost -- it was just converted into a different form.

%The gravitational and spring forces are conservative.
%Any work you do while lifting an object up into the air against the force 
%of gravity is not lost  it is \dokuitalic{stored} in the height of the object.
%You can get \dokuitalic{all} the work/energy back if you just let go of the 
%object. The energy will come back in the form of kinetic energy as the object
%gets accelerated during the fall.

%The negative of the work done against conservative forces is called \dokuitalic{potential energy}.
%Being high in the air means you have a lot of potential to fall,
%and compressing a spring by a certain distance means it has the 
%potential to spring back to its normal position.
%Let us look at the exact formulas now.

%
%\dokutitlelevelfour{Gravitational potential energy}

%The force of gravity is given by:
%\[
% F_g = W = mg.
%\]
%The gravitational potential energy of lifting an object for a height \(h\) is:
%\[
%  U_g = - \int_0^h \vec{F}_g \cdot d\vec{y} = \int_0^h mg dy = mh \int_0^h 1dy = mh y\big\vert_{y=0}^{y=h} = mgh.
%\]

%
%\dokutitlelevelfour{Spring energy}

%The force of a spring when stretched a distance \(x\) is given by:
%\[
% \vec{F}_s(\vec{x}) = - k\vec{x}.
%\]
%The sprint potential energy is:
%\[
% U_s = - \int_0^x \vec{F}_{s}(y) \cdot d\vec{y} = \int_0^x ky dy 
%= k\int_0^x y dy =  k\frac{1}{2}y^2\big\vert_{y=0}^{y=x} = \frac{1}{2}kx^2.
%\]

%
%\dokutitlelevelfour{Conservation of energy}

%Energy cannot be created or destroyed. 
%It just transforms from one form into another.

%If there are no dissipative forces, then we have
%\[
%  \sum E_i    =   \sum E_f.
%\]

%If there are dissipative forces like friction or external
%forces that do work, we must take their energy contributions into
%account as well:

%\[
% \sum E_i \  +\   W_{in}   =  \sum E_f,
%\]
%or
%\[
% \sum E_i    =  \sum E_f \  +\  W_{out}.
%\]

%This is one of the most important equations you will find in this book,
%because it will allow you to solve very complicated problems simply by
%accounting for all different kinds of energies.

%	

%	\dokutitleleveltwo{Uniform circular motion}
%	\label{42333dad828328e1c0baeb79429a3185}%% uniform_circular_motion

%	When a car makes a long left turn, the passenger riding shotgun will feel pushed
%	towards the right and into the door. 
%	If we assume the car moves at a constant speed \(v\) in the turn, and that
%	the radius of the curve is \(R\), what is the force of contact between the 
%	passenger and the door?
%	This question may sound complicated, but actually it boils down to a single formula.
%	This entire section is dedicated to that formula and to objects
%	moving around in a circle in general.

%	
%	\dokutitleleveltree{Concepts}
%	\label{ff4e01de0bd379280a0157bd102cc5f0}%% concepts

%	\begin{itemize}
%	\dokuitem  \(\hat{x},\hat{y}\): the a usual coordinate system.
%	\end{itemize}

%	In this section we will use a new type of coordinate system:

%	
%	\begin{itemize}
%	\dokuitem  \(\hat{r},\hat{t}\): the \dokuitalic{radial} and \dokuitalic{tangential} directions of a circular coordinate system. No matter where you are on the circle, the radial direction always points towards the centre, while the tangential direction is always perpendicular. On a bicycle wheel, the spokes are in the \(\hat{r}\) direction, while the pavement is in the \(\hat{t}\) direction.
%	\dokuitem  \(\vec{v}=(v_x,v_y)_{\hat{x}\hat{y}}=(v_r,v_t)_{\hat{r}\hat{t}}\): Velocity of particle, can be expressed as in \(xy\) or \(rt\) coordinates
%	\dokuitem  \(\vec{a}=(a_r,a_t)_{\hat{r}\hat{t}}\): the \dokuitalic{acceleration} of the particle.
%	\end{itemize}

%	Every time you have uniform circular motion (\(v_{t}=const\)), 
%	these variables will be related:

%	
%	\begin{itemize}
%	\dokuitem  \(R\): Radius of the circle of motion.
%	\dokuitem  \(v_t\): Speed of the circular motion in \([m/s]\). Sometimes referred to as \dokuitalic{tangential} speed.
%	\dokuitem  \(a_r\): Radial acceleration. The relation is \(a_r=\frac{v_t^2}{R}\).
%	\end{itemize}

%	There is some special terminology used to describe circular motion:

%	
%	\begin{itemize}
%	\dokuitem  \(C\): the \dokuitalic{circumference} of the circle of motion. For a circle of radius \(R\), \(C=2\pi R\).
%	\dokuitem  \(T\): the \dokuitalic{period}, how long it takes for the object to complete one circle. Measured in seconds.
%	\dokuitem  \(f\): the frequency of rotation. How many times per second does an object pass by the same point on the circle. \(f=\frac{1}{T}\). Measured in Hertz [Hz]=[1/s].
%	\dokuitem  \(\omega\equiv\frac{v_t}{R}\): angular velocity, how fast the angle of the object is rotating \(\omega=2\pi f\). 
%	\dokuitem  \(RPM\): the \dokuitalic{revolutions per minute} which is the angular velocity expressed in units of revolutions (1[rev]=$2\pi$[rad]) and minutes (1[min]=60[s]).
%	\end{itemize}

%	The angular velocity \(\omega\) is very useful because it describes the speed of 
%	circular motion of \dokuitalic{any} radius. 
%	Indeed, different points on a rotating disk will have different tangential speeds \(v_t\)
%	depending on how far from the radius they are, so it is much better to describe the
%	angular velocity and multiply by the radius as needed.

%	
%	\dokutitleleveltree{Formulas}
%	\label{51d24e1edefe34e683025dbba5c6eed6}%% formulas

%	\dokutitlelevelfour{Three directions}

%	Let's freeze time and zoom in on an object moving in a circle.
%	We can draw a force diagram with three directions:

%	
%	\begin{enumerate}\dokuitem  \(\hat{r}\): towards the centre of the circle of rotation.
%	\dokuitem  \(\hat{t}\): in the instantaneous direction where the car is moving right now. This is called the \dokuitalic{tangential} direction, from the greek to touch (imagine a straight line ``touching'' the circle).
%	\dokuitem  \(\hat{y}\): if necessary, we imagine a third \dokuitalic{vertical} axis pointing up out of the plane of rotation.
%	\end{enumerate}

%	\dokutitlelevelfour{Motion in a circle}

%	A circle of radius \(R\) has circumference:
%	\[
%	 C = 2 \pi R.
%	\]
%	An object moving along this circular path with
%	tangential speed \(v_t\).
%	The period, \(T\), is defined as how long it will take
%	the object to complete one turn and is equal to:
%	\[
%	 T = \frac{2\pi R}{v_t} = \frac{2\pi}{\omega} = \frac{1}{f}.
%	\]

%	
%	%\dokutitlelevelfour{Radial acceleration}
%	%\begin{wrapfigure}{r}{0pt}
%	%\includegraphics[width=125pt]{/Library/WebServer/Documents/miniref/lib/plugins/tmp/texitimages/texitJpeFEm.png}
%	%\end{wrapfigure}

%	The one equation of this section is this:
%	\[
%	  a_r = \frac{v^2_t}{ R },
%	\]
%	and it relates the \dokuitalic{radial} acceleration necessary to keep
%	an object turning in a circle of radius \(R\) at constant speed \(v_t\).

%	In particular, knowing \(v_t\) and \(R\) we can fill-in the right
%	hand side of the equation \(F=ma\) in the \(r\) direction:
%	\[
%	  \sum \vec{F}_{r} = ma_r = m \frac{v^2_t}{ R }.
%	\]
%	This means that we can solve for \(F_r\),
%	which could be due to a rope pulling towards the centre, of a tire-road friction force \(F_f\)
%	or some other force acting towards the centre.

%	%
%	%\dokutitleleveltree{Explanations}
%	%\label{678d5f6d14642c24a1e4bceffedbe407}%% explanations

%	%\dokutitlelevelfour{Radial coordinate system}

%	%To understand this whole \(\hat{r}\) and \(\hat{t}\) business, I want you to imagine for a
%	%second that you can freeze time. The \(\hat{r}\) direction will be towards the centre
%	%of the circle and \(\hat{t}\) is along a line touching the circle,  i.e., though moving 
%	%in a circle overall, instantaneously you are moving in the \(\hat{t}\) direction.
%	%If you unfreeze time and one microsecond elapses the \(\hat{r}\) and \(\hat{t}\) directions
%	%will change, you will have moved a little along the circle and the \(\hat{r}\) vector
%	%will have to rotate to keep pointing at the centre. Similarly, your instantaneous direction of motion, \(\hat{t}\), will have shifted a bit. 

%	%
%	%\dokutitlelevelfour{Period}

%	%From kinematics we know that motion with constant velocity (UVM)
%	%is described by the formula \(x = vt\), where the speed is 
%	%\dokuitalic{defined} as ratio of distance travelled over time: \(v=\frac{\Delta x}{\Delta t}\).
%	%In this case the distance travelled, \(\Delta x\), is not a straight line 
%	%but a curve following the circumference of a circle:
%	%\[
%	% C = 2\pi R.
%	%\]
%	%The period is defined as the time it takes for the object to complete one full turn:
%	%\[
%	% T = \frac{\text{distance}}{\text{speed}} = \frac{2\pi R}{v_t}.
%	%\]

%	%Tangential speed \(v_t\) is measured in \([m/s]\), whereas \(R\) is measured in \([m]\)
%	%so if you divide the two you get seconds \([s]\).

%	%
%	%\dokutitlelevelfour{Radial acceleration}

%	%Newton's first law says that if there is no acceleration acting on a moving object,
%	%then it will continue moving (1) in a straight line and (2) maintain the same speed.
%	%So far we have studied mostly problems involving \dokuitalic{linear acceleration}, 
%	%where some $a_x$ is causing changes in speed.
%	%This section deals with problems where the speed $\|\vec{v}\|$ 
%	%stays constant, but $\vec{v}$ changes direction.

%	%Circular motion violates the requirement (1) in Newton's law (conservation of momentum). 
%	%The logic is as follows. If you see any motion that is \dokuitalic{not a straight line}, 
%	%then there must be some acceleration involved which is producing the rotation.
%	%This is the radial acceleration.

%	%
%	%\dokutitlelevelfour{Centrifugal force}

%	%People often get confused about the meaning of things when they talk
%	%about circular motion. There is the point of view of the passenger,
%	% and there is the point of view of the car door, 
%	%which is holding the passenger into the circular trajectory.

%	%

%	%\begin{itemize}
%	%\dokuitem  \(\vec{F}_{dp}\): The amount of force the door is exerting on the passenger.
%	%\dokuitem  \(\vec{F}_{pd}\): The amount of force the passenger is exerting on the door.
%	%\end{itemize}

%	%This is the same force really, but one is relevant in the sum-of-the-forces
%	%calculation for the passenger \(\sum F_{psngr}\), whereas the second should be 
%	%added with all the other forces that act on the car (like the friction 
%	%between the pavement and the car tires).

%	%Viewed from the point of view of the passenger there is no confusion.
%	%The force \(F_{dp}\) acting towards the centre of rotation is causing
%	%a \(ma_r=m\frac{v_t^2}{R}\), and thus the circular motion 
%	%of the passenger results.

%	%Viewed from the point of view of the door, it feels a 
%	%certain push towards the outside of the circle which corresponds to the counter-force 
%	%to the force that the passenger feels.
%	%This comes from Newton's third law, which says that for each contact force 
%	%\(C_{12}\) exerted by object 1 on object 2, there exists a counter force 
%	%exerted of equal magnitude and opposite direction, \(C_{21}\) which is the
%	%force of object 2 pushing back.

%	%The confusion people get into, and the whole \dokuitalic{false concept} of centrifugal force,
%	%comes from thinking of a force diagram on yourself and falsely including the 
%	%outwards force on the door. You have to be clear. Which force diagram
%	%are you doing? If you are thinking of yourself as the object, the force that you
%	%feel is a \dokuitalic{centripetal} force. The door pushes you each instant towards the
%	%centre of the circle. If it weren't for the door, you would fly straight on.

%	%
%	%In some problems, it is also necessary to consider the \(y\) direction of the force diagram,
%	%where you will have the weight of the passenger \(\vec{W}=-mg\hat{\jmath}\).
%	%You should be able to solve joint \(ry\)-dimension problems, 
%	%with radial and vertical forces \(F_r\), \(F_y\) and radial and vertical 
%	%accelerations \(a_r\), \(a_y\).

%	%

%	\dokutitleleveltwo{Angular motion}
%	\label{318857367a8b72161a4bda9b7e83ce98}%% angular_motion

%	We will now study the physics of rotating objects.
%	Rotating disks, wheels, spinning footballs and ice skaters.
%	Anything spinning really.

%	%
%	%This is also a review chapter.
%	%You already know all the basic concepts for linear motion:
%	%position, velocity, acceleration, force, momentum and energy.
%	%In this chapter we define the \dokuitalic{angular} equivalents for 
%	%each of these, which will be appropriate whenever we have a rotating object.

%	
%	\dokutitleleveltree{Concepts}
%	\label{ff4e01de0bd379280a0157bd102cc5f0}%% concepts

%	Angular force:

%	
%	\begin{itemize}
%	\dokuitem  \(\mathcal{T}\): the \dokuitalic{torque} is rotational force. Measured in newton-meters [Nm].
%	\end{itemize}

%	Angular mass:

%	
%	\begin{itemize}
%	\dokuitem  \(I\): is the \dokuitalic{moment of inertia} of an object, and tells you how difficult it is to make it turn. Measured in [kg m\^{ }2].
%	\end{itemize}

%	Angular F=ma:

%	
%	\begin{itemize}
%	\dokuitem  \(\sum\mathcal{T}=I\alpha\): tells us that angular acceleration (\(\alpha\))  is caused by angular forces (torques) and the constant of proportionality  is the moment of inertia  which takes into account the mass of an object,  but also its shape. 
%	\end{itemize}

%	Angular motion (kinematics):

%	
%	\begin{itemize}
%	\dokuitem  \(\theta\equiv\frac{x_t}{R}\): angular displacement. Measured in radians [rad].
%	\dokuitem  \(\omega\equiv\frac{v_t}{R}\): angular velocity. Measured in [rad/s] or [RPM].
%	\dokuitem  \(\alpha\equiv\frac{a_t}{R}\): angular acceleration. Measured in [rad/s\^{ }2].
%	\end{itemize}

%	Angular momentum:

%	
%	\begin{itemize}
%	\dokuitem  \(L=I\omega\): the \dokuitalic{angular momentum} of a spinning object. Measured in [kg m\^{ }2/s].
%	\end{itemize}

%	Angular kinetic energy:

%	
%	\begin{itemize}
%	\dokuitem  \(K_r=\frac{1}{2}I\omega^2\): The \dokuitalic{angular kinetic energy} of a  spinning object. Measured in joules [J]=[kg m\^{ }2/s\^{ }2].
%	\end{itemize}

%	\dokutitleleveltree{Formulas}
%	\label{51d24e1edefe34e683025dbba5c6eed6}%% formulas

%	%\dokutitlelevelfour{Torque}
%	%\begin{wrapfigure}{r}{0pt}
%	%\includegraphics[width=150pt]{/Library/WebServer/Documents/miniref/lib/plugins/tmp/texitimages/texitb5N2t4.png}
%	%\end{wrapfigure}

%	Torque is defined as the rotational tendency of a force:
%	\[
%	 \mathcal{T} = F_{\perp}\times |r|= |F||r|\sin\theta.
%	\]
%	Note that only the perpendicular part of the force \(F_{\perp}\) is helping to cause a rotation.

%	To understand the meaning of the torque equation, you should stop reading and go
%	experiment with a regular door. If you push close to the hinges, it will take a lot 
%	more force to produce the same torque since the \(r\) will be small.
%	Also, if you grab the handles on both sides and pull away from the hinge,
%	your entire force will be \(F_{||}\), so no matter how hard you pull
%	you will cause zero torque.

%	The predominant convention is to call torques that produce counter-clockwise 
%	motion positive and torques that cause clockwise rotation negative.

%	
%	\dokutitlelevelfour{Torques cause angular acceleration}

%	The same way we have forces that cause acceleration (\(F=ma\)),
%	we have torques that cause angular acceleration:

%	\[
%	 \sum \mathcal{T} = \mathcal{T}_{net} = I \alpha.
%	\]

%	
%	\dokutitlelevelfour{Moment of inertia}

%	\[
%	  I = \{ \text{ how difficult it is to make an object turn } \}
%	\]
%	\[
%	 I_{disk} = \frac{1}{2}mR^2, \quad I_{ring}=mR^2, 
%	\]
%	\[
%	 I_{sphere} = \frac{2}{5} mR^2, \quad I_{sph. shell} = \frac{2}{3} mR^2.
%	\]

%	
%	\dokutitlelevelfour{Angular kinematics}

%	Instead of talking about position \(x\), we will now talk about
%	the angular orientation \(\theta\).
%	Instead of talking about velocity \(v\), we will talk about the
%	angular velocity \(\omega\).
%	Instead of acceleration, we have angular acceleration \(\alpha\).

%	Except for these change of actors, the \dokuitalic{play} remains the same.
%	Just like in the linear case, there are three equations
%	that fully describe uniform accelerated angular motion:
%	\[
%	  \alpha(t) = \alpha,
%	\]
%	\[
%	  \omega(t) = \alpha t + \omega_0,
%	\]
%	\[
%	  \theta(t) = \frac{1}{2}\alpha t^2 + \omega_0t + \theta_0,
%	\]
%	\[
%	 \omega_f^2 = \omega_0^2 + 2\alpha\Delta\theta.
%	\]

%	A special case of these equations is when there is no net torque
%	acting on the object. No torque means there will be no angular acceleration,
%	so the equations become a little simpler:
%	\[
%	  \alpha(t) = 0,
%	\]
%	\[
%	  \omega(t) = \omega_0,
%	\]
%	\[
%	  \theta(t) = \omega_0t + \theta_0.
%	\]
%	We call that \dokuitalic{uniform velocity angular motion}.

%	
%	\dokutitlelevelfour{Angular momentum}

%	The angular momentum of a spinning object is given by:
%	\[
%	 L = I \omega.
%	\]
%	The angular momentum of an object is a conserved quantity in the absence of torque:
%	\[
%	 L_{in} = L_{out}.
%	\]
%	This is similar to the way momentum \(\vec{p}\) is a conserved quantity 
%	in the absence of external forces \(\vec{F}_{net}\).

%	
%	\dokutitlelevelfour{Rotational kinetic energy}
%	\[
%	 K_r = \frac{1}{2} I \omega^2,
%	\]
%	which is the rotational analogue to the linear kinetic energy \(\frac{1}{2}mv^2\).

%	
%	\dokutitleleveltree{Static equilibrium II}
%	\label{e72ff188b85a8b9f85fd747dcd2decbf}%% static_equilibrium_ii

%	Equilibrium is the situation where \(\vec{F}_{net}=0\).
%	The forces on the objects balance each other, so there is no acceleration.

%	Conversely if you see an object that is not moving, then the forces
%	on it must be in equilibrium. There must be zero net force on it.
%	No net \(x\)-force:
%	\[
%	 \sum F_x = 0,
%	\]
%	no net \(y\)-force on it
%	\[
%	 \sum F_y = 0,
%	\]
%	and (this is the new part), if you see that it is not
%	rotating, then it must also have no net toques on it:
%	\[
%	 \sum \mathcal{T}_{(A)} = 0,
%	\]
%	where we can take the torques with respect to \dokuitalic{any} centre of rotation (A).

%	%
%	%\dokutitlelevelfour{Example: walking the plank}

%	%Plank is one third of the way out, and not attached to the ship.
%	%It weights 300 kg in total, so 100kg sticking out of the ship
%	%and 200kg on the ship.
%	%How far on the plank can you walk before it tips into the sea?

%	%
%	%\dokutitleleveltree{Explanations}
%	%\label{678d5f6d14642c24a1e4bceffedbe407}%% explanations

%	%\dokutitlelevelfour{Torque}

%	%People in the Mid West and other rural areas like to buy trucks because they have a lot of torque.
%	%Big engine, big torque. You can pull stuff. Wait isn't that force? But toque should be good too?
%	%What is the difference?  I am confused.

%	%Don't be confused. Torque is just force times distance, i.e. we have to take into account not only 
%	%what force is exerted but also the leverage: how far from the center of rotation you are exerting the force:
%	%\[
%	% \mathcal{T} = F_{rot} \times r.
%	%\]
%	%The bigger the leverage \(r\), the bigger torque you will create with a fixed amount of force.

%	%When it comes down to the question how much you can pull, and you want to answer it by
%	%looking at the specs of your truck, you have to look at (1) the max torque \(\mathcal{T}\) and (2) the radius of
%	%your wheels \(r\), and divide the numbers to get the maximum pulling force \(F\).

%	%
%	%\dokutitlelevelfour{Rotational F=ma}

%	%The angular analogue of $F=ma$ is:
%	%\[
%	% \mathcal{T} = I \alpha,
%	%\]
%	%where
%	%\[
%	% I = \sum m_i r_i^2 = \int_{obj} r^2 \ dm.
%	%\]

%	%
%	%\dokutitleleveltree{Examples}
%	%\label{bfebe34154a0dfd9fc7b447fc9ed74e9}%% examples

%	%\dokutitlelevelfour{Rotational UVM}

%	%Something is spinning at a constant angular velocity during
%	%some time \(t\). Just use \(\theta(t) = \omega t + \theta_0\).
%	%Waaay too simple and boring.

%	%
%	%\dokutitlelevelfour{Rotational UAM}

%	%A pulley of radius \(R\) and moment of inertia \(I\) has a rope wound around it
%	%and a mass \(m\) attached at the end of that rope.
%	%What is the angular acceleration caused by the mass
%	%as it unwinds the rope.

%	%The force diagram on \(m\) tells us that \(mg-T=ma_y\) (where \(\hat{y}\) points downwards).
%	%The torque diagram on the disk tells us that \(TR = I \alpha\).
%	%Adding \(R\) times the first equation to the second we get:
%	%\[
%	% R({mg - T}) + T  R  = R m a_y + I \alpha,
%	%\]
%	%or after simplification we get:
%	%\[
%	% R  m  g = R m a_y + I \alpha.
%	%\]
%	%But we know that the rope forms a solid connection between the disk and the mass block,
%	%so we must also have \(R \alpha = a_y\), so if we substitute for \(a_y\) we get:
%	%\[
%	% R  m  g = R m R \alpha + I \alpha = (R^2 m + I) \alpha,
%	%\]
%	%or 
%	%\[
%	% \alpha = \frac{  R  m  g  }{ R^2 m + I }.
%	%\]
%	%This makes sense. The numerator is the ``cause'', and the denominator is the effective moment of inertia of the system
%	%as a whole.

%	%\dokutitleleveltree{Discussion}
%	%\label{bd8bc36eb41bc90c585ae7e902e9e284}%% discussion

%	%It is really important that you connect all of the above 
%	%rotational concepts with their linear counterparts.

%	%

%	
%	\dokutitleleveltwo{Simple harmonic motion}
%	\label{cd80db888044ca76c4e83c74b55b1b83}%% simple_harmonic_motion

%	This law describes the motion of a mass attached to a spring, a pendulum and any system that goes back and forth in a cyclic fashion.

%	
%	\dokutitleleveltree{Concepts}
%	\label{ff4e01de0bd379280a0157bd102cc5f0}%% concepts

%	\begin{itemize}
%	\dokuitem  \(A\): Amplitude of the movement, how far does the object go back and forth.
%	\dokuitem  \(t\): time.
%	\dokuitem  \(x(t)\): position of object at time \(t\).
%	\dokuitem  \(\omega\): angular frequency.
%	\dokuitem  \(\phi\): phase constant.
%	\dokuitem  \((\omega t + \phi)=\theta\): phase, the argument of the function \(\sin\).
%	\dokuitem  \(T\): the \dokuitalic{period} is the time it takes for the movement to repeat. Measured in seconds [s].
%	\dokuitem  \(f\): frequency [Hz]=[1/s].
%	\end{itemize}

%	\dokutitleleveltree{Formulas}
%	\label{51d24e1edefe34e683025dbba5c6eed6}%% formulas
%	Frequency is defined as ``how many cycles in one second'' and is
%	equal to the inverse of the period (how long one cycle takes):
%	\[
%	 f=1/T=\frac{\omega}{2\pi} \text{ [Hz].}
%	\]
%	The relation between \(f\) (frequency) and \(\omega\) (angular frequency)
%	is a multiplication by \(2\pi\) needed to match the units:
%	1 cycle needs to be multiplied by the number of radians it takes to 
%	make one full turn.

%	The equation of motion of an object undergoing simple harmonic motion is:
%	\[
%	 x(t)=A\cos(\omega t   + \phi).
%	\]
%	%
%	%Now don't be scared. It is really simple.
%	%Let us just study the properties of \(\sin(x)\) for a moment,
%	%so that you can be get familiar with the math of periodic functions.

%	%
%	%\dokutitlelevelfour{The function sin(x)}

%	%\begin{wrapfigure}{r}{0pt}
%	%\includegraphics[width=165pt]{/Library/WebServer/Documents/miniref/lib/plugins/tmp/texitimages/texitmrNnmQ.png}
%	%\end{wrapfigure}

%	%
%	%The functions \(f(x)=\sin(x)\) and \(f(x)=\cos(x)\) are both angular functions,
%	%that is, they take an angle \(\theta\) as input.
%	%Sometimes we call the input \(x\), but in the next paragraphs
%	%we will call the input \(\theta\) so that we keep in mind
%	%that we are talking about angles.

%	%
%	%The period \(T\) of \(\sin(t)\) with no coefficients
%	%inside of it is \(2\pi\), because this is how long
%	%it takes (in radians) to do one full turn.

%	%
%	%\dokutitlelevelfour{Time-scaling sin}

%	%If you want a sin function with a different period,
%	%you have to add a multiplier in front of \(x\) inside
%	%the sin. This multiplier is, by convention, called
%	%\(\omega\), so now we have a input-scaled \(\sin\) function:
%	%\[
%	% \sin(\omega t ),
%	%\]
%	%which will have period
%	%\[
%	% T=\frac{2\pi}{\omega}.
%	%\]
%	%No seriously, if you don't believe me check for yourself.
%	%If you let \(t\) go from \(0\) to \(T\) you should see
%	%the motion go though one cycle. 
%	%Indeed, when multiplied with the scaling \(\omega=\frac{2\pi}{T}\),
%	%the input to sin goes from \(0\) to \(2\pi\) as \(t\) goes from \(0\) to \(T\).

%	%
%	%\dokutitlelevelfour{Sin or cos up to you}

%	%Simple harmonic motion is equally well described
%	%by the \(\sin\) function or the \(\cos\) function.

%	%You should use cos when the spring starts from the stretched position: 
%	%\(x_i = x_{max}=A\), because naturally when \(t=0\) cos
%	%will be one, and the overall function will be:
%	%\[
%	% x(t) = A\cos(\omega t).
%	%\]

%	%If the harmonic motion starts from \(x(t)=0\)
%	%when \(t=0\), then \(\sin\) is a more appropriate
%	%function to describe this motion:
%	%\[
%	% x(t) = A\sin(\omega t),
%	%\]
%	%which correctly predicts the initial position \(x(0)=0\).

%	%Any combination of the above scenarios is possible too.
%	%So now it becomes difficult to see what the motion is,
%	%with \dokuitalic{two} amplitudes (one for stretched initial conditions,
%	%and one for centered  initial conditions) and a sum of 
%	%sin and cos:
%	%\[
%	% x(t) = A_1\cos(\omega t) + A_2\sin(\omega t).
%	%\]
%	%
%	%Lo and behold. There is actually a simpler way of 
%	%writing the above equation. Sin and cos are 
%	%identical except for the fact that one is a
%	%shifted version of the other.
%	%By using trigonometric identity Kung Fu, we can 
%	%rewrite the above in terms of sin or cos with some 
%	%combined amplitude and a \dokuitalic{phase shift}:
%	%\[
%	% x(t)=A\cos(\omega t   + \phi).
%	%\]
%	%The initial conditions $x_i$ and $v_i$ are used 
%	%to find either the individual coefficients $A_1$ and $A_2$
%	%of the sin and the cos or the common coefficient $A$ and
%	%the phase shift $\phi$.

%	%Let me go over what just happened here one more time,
%	%because this is some crazy stuff.
%	%You are not supposed to learn about this stuff until 
%	%the \dokuitalic{differential equations} class, but I will tell
%	%you about it right now because it is like \dokuitalic{woooow}.
%	%Let's draw an analogy with a situation which we have
%	%seen previously. We are trying to describe (and predict)
%	%the position of an object as a function of time \(x(t)\).
%	%In kinematics there was \dokuitalic{one} equation \(x(t)=x_i+v_it + \frac{1}{2}at^2\),
%	%and depending on the initial velocity and the initial position we got 
%	%different trajectories, because of the different values of \(x_i\)
%	%and \(v_i\).
%	%In the case of simple harmonic motion we have \dokuitalic{two} functions and
%	%the initial position \(x_i\) and initial velocity \(v_i\) determine
%	%how much of each function we should include
%	%in the formula for \(x(t)\).

%	%The thing to remember is that you can pick cos or sin 
%	%as the situation requires and, if the situation requires
%	%it, add a phase factor.
%	%I assure you that the fact that \(a\sin(x)+b\cos(x)\) can
%	%be expressed as \(A\cos(x+\phi)\) or \(A\sin(x-\psi)\) is not 
%	%magic, but simple application of \hyperref[dfc96c1062532291f8378374353a33c7]{trig{\textbackslash}\_identities}.

%	
%	\dokutitlelevelfour{Mass and spring}

%	Suppose you have a mass \(m\) attached
%	to a spring with spring constant \(k\).
%	If disturbed from rest, this mass-spring
%	system will undergo simple harmonic motion
%	with angular frequency:
%	\[
%	 \omega = \sqrt{ \frac{k}{m} }.
%	\]
%	%
%	%A typical exam question is to tell you \(k\) and \(m\)
%	%and ask about the period \(T\), at which point you have
%	%to remember the definition of period to obtain the 
%	%answer:
%	%\[
%	% T = \frac{2\pi}{\omega} =  2\pi \sqrt{ \frac{m}{k} }.
%	%\]

%	
%	\dokutitlelevelfour{Pendulum}

%	Consider an object suspended at the end of a long string 
%	of length \(\ell\) in a gravitational field of strength \(g\).
%	If disturbed from equilibrium this system will undergo
%	simple harmonic motion, by swinging from side to side.

%	The period of oscillation will be:
%	\[
%	 T = 2\pi \sqrt{ \frac{\ell}{g} }.
%	\]
%	%Note in particular that this equation does not depend
%	%on the amplitude of the oscillation (how far the pendulum swings)
%	%nor the mass of the pendulum. Kind of cool no?
%	%For a long while, this was how people kept track of time: with pendulum clocks.

%	%To describe the angular position of the pendulum 
%	%we can use a the formula:
%	%\[
%	% \theta(t) = \theta_{max} \cos(\omega t + \phi),
%	%\]
%	%where \(\theta_{max}\) is the maximum angle the pendulum swings to.
%	%The coefficient inside the \(\cos\) is equal to:
%	%\[
%	% \omega \equiv \frac{2\pi}{T} = \sqrt{ \frac{g}{\ell} }.
%	%\]

%	
%	\dokutitlelevelfour{SHM equations of motion}

%	These are the new equations of motion which you need
%	to learn how to use:

%	\begin{align*}
%	x(t) &= A\cos(\omega t + \phi), \\
%	v(t) &= -A\omega \sin(\omega t + \phi), \\
%	a(t) &= -A\omega^2\cos(\omega t + \phi).
%	\end{align*}

%	Note that the velocity and the acceleration of the object
%	are also periodic functions. 
%	Pay attention to the \dokuitalic{amplitude} of the velocity:
%	\[
%	 v_{max} = \omega A,
%	\]
%	and the \dokuitalic{amplitude} of the acceleration:
%	\[
%	 a_{max} = \omega^2 A.
%	\]
%	You will often be asked to solve for these quantities,
%	which is going to be an easy task if you know \(A\) and \(\omega\).

%	%
%	%And equivalent set of equations of motion exists with \(x(t)\propto \sin(\omega t + \psi)\),
%	%but in this section we will just talk about the \(\cos\) kind of \(x(t)\). 
%	%Everything I say would be true if we had
%	%started with \(x(t)\propto \sin(\omega t + \psi)\) instead.

%	
%	\dokutitlelevelfour{Energy}

%	The potential energy stored in the spring is: 
%	\[
%	 U(t)= \frac{1}{2} kx(t)^2 =\frac{1}{2}kA^2\cos^2(\omega t +\phi).
%	\]

%	
%	The kinetic energy of the mass is:
%	\[
%	 K(t)= \frac{1}{2} mv(t)^2 = \frac {1}{2}m\omega^2A^2\sin^2(\omega t +\phi).
%	\]

%	The total energy is: \\ 
%	\begin{align*}
%	 E_{total}
%	 &= U(t) + K(t) \\
%	 &= \frac{1}{2}kA^2\cos^2(\omega t) + \frac {1}{2}m\omega^2A^2\sin^2(\omega t) \\
%	 &= \frac{1}{2}kA^2\underbrace{\left[ \cos^2(\omega t) + \sin^2(\omega t)\right]}_{=1} =\frac{1}{2}kA^2 = U_{max} \\
%	% &= \frac{1}{2}m\omega^2A^2\cos^2(\omega t ) + \frac {1}{2}m\omega^2A^2\sin^2(\omega t ) \ \ (\text{since } k  = m\omega^2 )\\
%	 &=  %\frac{1}{2}m\underbrace{\omega^2A^2}_{v_{max}^2}\underbrace{\left[ \cos^2(\omega t) + \sin^2(\omega t)\right]}_{=1} 
%	  \frac{1}{2}mv_{max}^2 = K_{max}.
%	\end{align*}

%	%
%	%\dokutitlelevelfour{Conservation of energy}

%	%In simple harmonic motion, the total 
%	%energy is conserved:
%	%\[
%	% E_T = U+ K.
%	%\]
%	%Energy shifts between the potential energy of the spring
%	%and the kinetic energy of the moving mass.
%	%For any two points \(t_i\) and \(t_f\):
%	%\[
%	% U_i + K_i = U_f + K_f.
%	%\]

%	%Two particularly important moments of the SHM are
%	%the points where it is at its maximum position
%	%and zero velocity \(x=\pm A\), \(K=0\), \(E_T= U\), 
%	%which corresponds to all the energy being stored in the spring,
%	%and the point where it has zero displacement but
%	%maximal velocity \(x=0, U=0, v=\pm A\omega, E_T=K\),
%	%which corresponds to all the energy being kinetic.

%	%
%	%\dokutitleleveltree{Explanations}
%	%\label{678d5f6d14642c24a1e4bceffedbe407}%% explanations

%	%\dokutitlelevelfour{Derivation of SHM equation}

%	%OK, wait a minute, you are saying. 
%	%First I told you about uniform velocity \(x(t)=v_it+x_i\)
%	%and then I added in acceleration \(x(t) = x_i + v_it + \frac{1}{2}at^2\),
%	%which is a little more complex but still manageable.
%	%Now I am talking about \(\sin\) and \(\cos\) and Greek letters with dubious
%	%names like phase. 
%	%Are you phased by all of this?
%	%When I was learning this stuff, I was totally
%	%phased because I didn't understand where the \(\sin\) and \(\cos\) came from.

%	%The \(\sin\) comes from \(F=ma\),
%	%and the fact that the force of a spring is \(F_{s} = -kx\).

%	%Recall the definition of acceleration:
%	%\[
%	% a=\frac{dv}{dt} = \frac{d^2x(t)}{dt^2}.
%	%\]

%	%We now substitute these facts into \(F=ma\) to get:

%	%\begin{align*}
%	% F & =ma \\
%	% -kx  	&= ma 	\\
%	% -kx(t) &= m\frac{d^2x(t)}{dt^2} \\
%	% 0 & = m\frac{d^2x}{dt^2}+ kx(t) \\
%	% 0 & = \frac{d^2x}{dt^2}+ \frac{k}{m}x(t).
%	%\end{align*}

%	%This is as far as Newton's second law will be able to bring us.
%	%The formula \(F=ma\) is not enough 
%	%to tell us the exact function \(x(t)\), like in
%	%the UVM and UAM kinematics motion in the earlier
%	%chapters.  Instead of knowing the function \(x(t)\),
%	%we know one of its properties, namely, that its
%	%second derivative is equal to the negative of itself
%	%multiplied by some numbers.

%	%Can you think of a function whose second derivative
%	%is equal to itself times \(\frac{k}{m}\)? 
%	%Ok I thought of one:
%	%\[
%	% x_1(t)=A \cos( \sqrt{ \frac{k}{m}}t + \phi),
%	%\]
%	%and come to think of it I thought of a second
%	%one which also works:
%	%\[
%	% x_2(t)=A \sin( \sqrt{ \frac{k}{m}}t + \psi).
%	%\]
%	%Hmm, now you see where the \(\sin\) comes from 
%	%and why we said that \(\omega = \sqrt{ \frac{k}{m} }\)
%	%for a mass-spring system. It is all from \(F=ma\).

%	%
%	%\dokutitleleveltree{Examples}
%	%\label{bfebe34154a0dfd9fc7b447fc9ed74e9}%% examples

%	%In various word problems you will either be told the initial
%	%amplitude \(x_i=A\) or the initial velocity \(v_i=\omega A\)
%	%and asked to solve for some unknown. 
%	%Nothing should be too difficult provided
%	%you write down the equations 
%	%\(x(t)\), \(v(t)\) and \(a(t)\) and fill in the knowns.

%	%
%	%\dokutitleleveltree{Links}
%	%\label{807765384d9d5527da8848df14a4f02f}%% links

%	%



\section{Conclusion}

% Summary
The numbered equations...


\section{Minireference}


I hope this short excerpt from the \texttt{MATH and PHYSICS Minireference} 
has given you some inspiration for compact teaching. No blah blah. Straight to the point.

If you liked this tutorial you can check out the other ones on \url{http://minireference.com}
and order the printed book which has not only formulas but also compact explanations:
\url{http://minireference.com/order_book/}.

\end{document}
