
%!TEX root = mechanics_miniref_flyer.tex

To solve a physics problem is to obtain the \emph{equation of motion} $x(t)$, 
which describes the position of the object as a function of time.
%
Once you know $x(t)$, you can answer any question pertaining to the motion of the object.
To find the initial position $x_i$ of the object, you simply plug $t=0$ into the equation of motion $x_i = x(0)$.
To find the time(s) when the object reaches a distance of 20[m] from the origin, we simply solve for $t$ in $x(t)=20$[m].
Many of the problems on the final exam will be of this form so if know how to find $x(t)$,
you will be in good shape for the exam.

%\paragraph{{\bf Forces}}
\vspace{-3mm}
\subsection{Dynamics is the study of forces}
The first step towards finding $x(t)$ is to calculate all the \emph{forces} that act on the object.
Forces are the \emph{cause} of motion, so if you want to understand motion you need to understand forces. 
Newton's second law $F=ma$ states that {\bf a force acting on an object produces an acceleration}
inversely proportional to the mass of the object. 
Once you have the acceleration, you can compute $x(t)$ using calculus.
We will discuss the calculus procedure for getting from $a(t)$ to $x(t)$ shortly.
For now, let us focus on the causes of motion: the forces acting on the object.
There are many kinds of forces: the weight of an object $\vec{W}$ is a type force, 
the force of friction $\vec{F}_f$ is another type of force, the tension in a rope $\vec{T}$ is 
yet another type of force and there are many others.
Note the little arrow on top of each force, which is there to remind you that forces are \emph{vector} quantities.
Unlike regular numbers, forces act in a particular direction, so it is possible that the effects of 
one force are counteracting another force. For example the force of the weight of a flower pot 
is exactly counter-acted by the tension in the rope on which it is suspended, thus,
while there are two forces that may be acting on the pot, there is no \emph{net force} acting on it.
Since there is no net force to cause motion and since the pot wasn't moving to begin with, 
it will just sit there motionless despite the fact that there are forces acting on it!
The first step when analyzing a physics problem is to calculate the \emph{net force}
acting on the object, which is the sum of all the forces acting on the object $\vec{F}_{net} \equiv \sum \vec{F}$.
Knowing the net force, we can use $\vec{a}(t) = \frac{\vec{F}_{net}}{m}$ to find the acceleration.

\vspace{-3mm}
\subsection{Kinematics is the study of motion}
If you know the acceleration of an object $a(t)$ and its initial velocity $v_i$, 
you can find its velocity $v(t)$ function at all later times. 
This is because the acceleration function $a(t)$ describes the change in the velocity of the object.
If you know that the object started with an initial velocity of $v_i \equiv v(0)$,
the velocity at a later time $t=\tau$ is equal to $v_i$ plus the "total velocity change" between $t=0$ and $t=\tau$.
The mathematical way of saying this is $v(\tau)=v_i+\int_0^\tau a(t)\;dt$.
The symbol $\int \cdot \;dt$ is called an \emph{integral} and is fancy way of finding the total
of some quantity over a given time period. 
In the above formula we were calculating the total of $a(t)$ between $t=0$ and $t=\tau$.

To understand what is going on, it may be useful to draw an analogy with a scenario you are more familiar with.
Consider the function $\textrm{ba}(t)$ which represents your account balance at time $t$,
and the function $\textrm{tr}(t)$ which corresponds to the transactions (deposits and withdraws) on your account.
The function $\textrm{tr}(t)$ describes the change in the function $\textrm{ba}(t)$,
the same way the function $a(t)$ describes the change in $v(t)$.
Knowing the initial balance of your account at the beginning of the month,
you can calculate the balance at the end of the month as follows $\textrm{ba}(30)=$\textrm{ba}(0)$+\int_0^{30} \textrm{tr}(t)\:dt$.

If you know the initial position $x_i$ and the velocity function $v(t)$ you can find the position function $x(t)$ by using integration again.  
We find the position at time $t=\tau$ by adding up all the velocity (changes in the position). 
The formula is $x(\tau) = x_i + \int_0^\tau v(t)\:dt$.


The entire procedure for predicting the motion of objects can be summarized as follows:
\begin{equation}
\frac{1}{m} \underbrace{ \left( \sum \vec{F} = \vec{F}_{net}  \right) }_{\text{dynamics}} = \underbrace{ a(t) \ \overset{v_i+ \int\!dt }{\longrightarrow} \ v(t) \ \overset{x_i+ \int\!dt }{\longrightarrow} \ x(t) }_{\text{kinematics}}.
 \label{fma-eqn}
\end{equation}
%
If you understand the above equation, then you understand mechanics.
My goal for the next couple of pages is to introduce you to all the concepts that 
appear in that equation and the relationships between them. 




\vspace{-3mm}
\subsection{Other concepts}

Apart from equation \eqref{fma-eqn}, there is a number of other topics which are part of a standard Mechanics class. 

Newton's second law can also be applied to the study of circular motion.
Circular motion is described by the angle of rotation $\theta(t)$, 
the angular velocity $\omega(t)$ and the angular acceleration $\alpha(t)$.
The causes of angular acceleration are angular forces, which we call \emph{torques} $\mcal{T}$.
Apart from this change to angular quantities, the principles behind the circular motion 
are exactly the same as those for linear motion.
%and Newton's second law for rotational motion is $\mcal{T}_{net}=I\alpha$.
%We will 
%The quantity $I$ is called the \emph{moment of inertia} and describes how difficult it is to make an object turn.
%Circular motion is therefore 
%\begin{equation}
% \frac{1}{I}\sum \mathcal{T} = \alpha(t) \ \overset{\omega_i+ \int\!dt }{\longrightarrow} \ \omega(t) \ \overset{\theta_i+ \int\!dt }{\longrightarrow} \ \theta(t).
% \label{fma-eqn}
%\end{equation}


%\begin{equation}
%  U(h) = - \int_0^h  \vec{F}_g \cdot d\vec{y} = - \int_0^h  (- mg\hat{\jmath}) \cdot \hat{\jmath} \;dy    = mgh.
%  \nonumber
%\end{equation}



During a collision between two objects there will be a sudden spike in the contact force between them,
which can be difficult to measure and quantify.
It is therefore not possible to use Newton's law $F=ma$ to predict the accelerations that occur during collisions.
%and the resulting motion of the objects after they collide.
In order to predict the motion of the objects after the collision we must use a \emph{momentum} calculation.
An object of mass $m$ moving with velocity $\vec{v}$ has momentum $\vec{p}\equiv m\vec{v}$.
The principle of conservation of momentum states that {\bf the total amount of momentum before and
after the collision is conserved}. Thus, if two objects with initial momenta $\vec{p}_{i1}$ and $\vec{p}_{i2}$ collide,
the total momentum before the collision must be equal to the total momentum after the collision:
\[
	\sum \vec{p}_i = \sum \vec{p}_f 	
	\qquad
	\Rightarrow
	\qquad
	\vec{p}_{i1} + \vec{p}_{i2} 
		=
		\vec{p}_{f1} + \vec{p}_{f2}.
\]
Using this equation, it is possible to calculate the final momenta $\vec{p}_{f1}$, $\vec{p}_{f2}$
of the objects after the collision. 


Another way of solving physics problems is to use the concept of energy.
Instead of trying to describe the entire motion of the object, 
we can focus only on the initial parameters and the final parameters.
The law of conservation of energy states that {\bf the total energy of the system is conserved}.
Knowing the total initial energy of a system allows us to find final energy,
and from this calculate the final motion parameters.

\vspace{-3mm}
\subsection{Reality check}

Of course, you must realize that reading a \fourrr page tutorial on Mechanics will not make an expert out of you.
Mechanics expertise can only come from doing exercises on your own:
``Il faut souffrir pour \^etre boll\'e."
What we \emph{can} do in \fourrr pages is to go over all the important 
concepts and state the important formulas which connect the concepts.
%
There are two ways of looking at Mechanics: either as an opportunity to play {\sc lego} with scientific building blocks,
or as a horrible chore inflicted upon requiring complicated mathematical prerequisites.
The choice is up to you.

Speaking of prerequisites, I want to reassure you that you 
have nothing to worry about on that front.
The hardest math you will have to do is solving a quadratic equation.
We will cover everything you need to know about vectors and integrals
in the next section.

